\chapter{Functions}

\section{Extensions}
\label{sec:fun-extensions}

\begin{function}{ensure-function}{function-designator}{alexandria}{}
  Returns the function designated by FUNCTION-DESIGNATOR:
if FUNCTION-DESIGNATOR is a function, it is returned, otherwise
it must be a function name and its FDEFINITION is returned.
\end{function}

\begin{function}{disjoin}{predicate \rest more-predicates}{alexandria}{}
  Returns a function that applies each of PREDICATE and MORE-PREDICATE
functions in turn to its arguments, returning the primary value of the first
predicate that returns true, without calling the remaining predicates.
If none of the predicates returns true, NIL is returned.
\end{function}

\begin{function}{conjoin}{predicate \rest more-predicates}{alexandria}{}
  Returns a function that applies each of PREDICATE and MORE-PREDICATE
functions in turn to its arguments, returning NIL if any of the predicates
returns false, without calling the remaining predicates. If none of the
predicates returns false, returns the primary value of the last predicate.
\end{function}

\begin{function}{compose}{function \rest more-functions}{alexandria}{}
  Returns a function composed of FUNCTION and MORE-FUNCTIONS that applies its
arguments to to each in turn, starting from the rightmost of MORE-FUNCTIONS,
and then calling the next one with the primary value of the last.
\end{function}

\begin{function}{multiple-value-compose}{function \rest more-functions}{alexandria}{}
  Returns a function composed of FUNCTION and MORE-FUNCTIONS that applies
its arguments to to each in turn, starting from the rightmost of
MORE-FUNCTIONS, and then calling the next one with all the return values of
the last.
\end{function}

\begin{function}{curry}{function \rest arguments}{alexandria}{}
  Returns a function that applies ARGUMENTS and the arguments
it is called with to FUNCTION.
\end{function}

\begin{function}{rcurry}{function \rest arguments}{alexandria}{}
  Returns a function that applies the arguments it is called
with and ARGUMENTS to FUNCTION.
\end{function}

\begin{macro}{named-lambda}{name lambda-list \body body}{alexandria}{}
  Expands into a lambda-expression within whose BODY NAME denotes the
corresponding function.
\end{macro}
