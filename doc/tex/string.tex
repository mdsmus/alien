\chapter{Strings}

\begin{function}{parse-integer}{string \key start end radix junk-allowed}{}{}
  Examine the substring of string delimited by start and end
  (default to the beginning and end of the string)  It skips over
  whitespace characters and then tries to parse an integer. The
  radix parameter must be between 2 and 36.
\end{function}

\begin{accessor}{char}{string index}{}{}
  Given a string and a non-negative integer index less than the length of
  the string, returns the character object representing the character at
  that position in the string.
\end{accessor}

\begin{accessor}{schar}{string index}{}{}
  SCHAR returns the character object at an indexed position in a string
   just as CHAR does, except the string must be a simple-string.
\end{accessor}

\begin{function}{make-string}{count \key element-type (initial-element fill-char)}{}{}
  Given a character count and an optional fill character, makes and returns a
new string COUNT long filled with the fill character.
\end{function}

\begin{function}{nstring-capitalize}{string \key start end}{}{}
  
\end{function}

\begin{function}{nstring-downcase}{string \key start end}{}{}
  
\end{function}

\begin{function}{nstring-upcase}{string \key start end}{}{}
  
\end{function}

\begin{function}{simple-string-p}{object}{}{}
  Return true if OBJECT is a SIMPLE-STRING, and NIL otherwise.
\end{function}

\begin{function}{string}{x}{}{}
  Coerces X into a string. If X is a string, X is returned. If X is a
   symbol, X's pname is returned. If X is a character then a one element
   string containing that character is returned. If X cannot be coerced
   into a string, an error occurs.
\end{function}

\begin{function}{string-capitalize}{string \key start end}{}{}
  
\end{function}

\begin{function}{string-downcase}{string \key start end}{}{}
  
\end{function}

\begin{function}{string-equal}{string1 string2 \key start1 end1 start2 end2}{}{}
  Given two strings (string1 and string2), and optional integers start1,
  start2, end1 and end2, compares characters in string1 to characters in
  string2 (using char-equal).
\end{function}

\begin{function}{string-greaterp}{string1 string2 \key start1 end1 start2 end2}{}{}
  Given two strings, if the first string is lexicographically greater than
  the second string, returns the longest common prefix (using char-equal)
  of the two strings. Otherwise, returns ().
\end{function}

\begin{function}{string-left-trim}{char-bag string}{}{}
  
\end{function}

\begin{function}{string-lessp}{string1 string2 \key start1 end1 start2 end2}{}{}
  Given two strings, if the first string is lexicographically less than
  the second string, returns the longest common prefix (using char-equal)
  of the two strings. Otherwise, returns ().
\end{function}

\begin{function}{string-not-equal}{string1 string2 \key start1 end1 start2 end2}{}{}
  Given two strings, if the first string is not lexicographically equal
  to the second string, returns the longest common prefix (using char-equal)
  of the two strings. Otherwise, returns ().
\end{function}

\begin{function}{string-not-greaterp}{string1 string2 \key start1 end1 start2 end2}{}{}
  Given two strings, if the first string is lexicographically less than
  or equal to the second string, returns the longest common prefix
  (using char-equal) of the two strings. Otherwise, returns ().
\end{function}

\begin{function}{string-not-lessp}{string1 string2 \key start1 end1 start2 end2}{}{}
  Given two strings, if the first string is lexicographically greater
  than or equal to the second string, returns the longest common prefix
  (using char-equal) of the two strings. Otherwise, returns ().
\end{function}

\begin{function}{string-right-trim}{char-bag string}{}{}
  
\end{function}

\begin{function}{string-trim}{char-bag string}{}{}
  
\end{function}

\begin{function}{string-upcase}{string \key start end}{}{}
  
\end{function}

\begin{function}{string/=}{string1 string2 \key start1 end1 start2 end2}{}{}
  Given two strings, if the first string is not lexicographically equal
  to the second string, returns the longest common prefix (using char=)
  of the two strings. Otherwise, returns ().
\end{function}

\begin{function}{string<}{string1 string2 \key start1 end1 start2 end2}{}{}
  Given two strings, if the first string is lexicographically less than
  the second string, returns the longest common prefix (using char=)
  of the two strings. Otherwise, returns ().
\end{function}

\begin{function}{string<=}{string1 string2 \key start1 end1 start2 end2}{}{}
  Given two strings, if the first string is lexicographically less than
  or equal to the second string, returns the longest common prefix
  (using char=) of the two strings. Otherwise, returns ().
\end{function}

\begin{function}{string=}{string1 string2 \key start1 end1 start2 end2}{}{}
  Given two strings (string1 and string2), and optional integers start1,
  start2, end1 and end2, compares characters in string1 to characters in
  string2 (using char=).
\end{function}

\begin{function}{string>}{string1 string2 \key start1 end1 start2 end2}{}{}
  Given two strings, if the first string is lexicographically greater than
  the second string, returns the longest common prefix (using char=)
  of the two strings. Otherwise, returns ().
\end{function}

\begin{function}{string>=}{string1 string2 \key start1 end1 start2 end2}{}{}
  Given two strings, if the first string is lexicographically greater
  than or equal to the second string, returns the longest common prefix
  (using char=) of the two strings. Otherwise, returns ().
\end{function}

\begin{function}{stringp}{object}{}{}
  Return true if OBJECT is a STRING, and NIL otherwise.
\end{function}

\begin{class}{string}{x}{}{}
  Coerces X into a string. If X is a string, X is returned. If X is a
   symbol, X's pname is returned. If X is a character then a one element
   string containing that character is returned. If X cannot be coerced
   into a string, an error occurs.
\end{class}

\section{String extensions}
\label{sec:string-extensions}

\begin{function}{parse-float}{float-string \key start end radix junk-allowed type decimal-character}{arnesi}{}
\end{function}
