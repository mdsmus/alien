\chapter{Iteration}

\begin{macro}{do}{varlist endlist \body body}
  DO ({(Var [Init] [Step])}*) (Test Exit-Form*) Declaration* Form*
  Iteration construct. Each Var is initialized in parallel to the value of the
  specified Init form. On subsequent iterations, the Vars are assigned the
  value of the Step form (if any) in parallel. The Test is evaluated before
  each evaluation of the body Forms. When the Test is true, the Exit-Forms
  are evaluated as a PROGN, with the result being the value of the DO. A block
  named NIL is established around the entire expansion, allowing RETURN to be
  used as an alternate exit mechanism.
\end{macro}

\begin{macro}{do*}{varlist endlist \body body}
  DO* ({(Var [Init] [Step])}*) (Test Exit-Form*) Declaration* Form*
  Iteration construct. Each Var is initialized sequentially (like LET*) to the
  value of the specified Init form. On subsequent iterations, the Vars are
  sequentially assigned the value of the Step form (if any). The Test is
  evaluated before each evaluation of the body Forms. When the Test is true,
  the Exit-Forms are evaluated as a PROGN, with the result being the value
  of the DO. A block named NIL is established around the entire expansion,
  allowing RETURN to be used as an laternate exit mechanism.
\end{macro}

\begin{macro}{dolist}{var \body body \env env}
  
\end{macro}

\begin{macro}{dotimes}{var \body body}
  
\end{macro}

\begin{macro}{loop}{\env env \rest keywords-and-forms}
  
\end{macro}

\section{Iteration extensions}
\label{sec:iteration-extensions}

See also \fun{do-range}
