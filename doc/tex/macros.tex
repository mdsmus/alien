
\section{Macros}
\label{sec:macros}

\begin{macro}{with-gensyms}{names \body forms}{alexandria}{}
  Binds each variable named by a symbol in NAMES to a unique symbol around
FORMS. Each of NAMES must either be either a symbol, or of the form:

 (symbol string-designator)

Bare symbols appearing in NAMES are equivalent to:

 (symbol symbol)

The string-designator is used as the argument to GENSYM when constructing the
unique symbol the named variable will be bound to.
\end{macro}

\begin{macro}{with-unique-names}{names \body forms}{alexandria}{}
  Alias for WITH-GENSYMS.
\end{macro}

\begin{macro}{once-only}{specs \body forms}{alexandria}{}
  Each SPEC must be either a NAME, or a (NAME INITFORM), with plain
NAME using the named variable as initform.

Evaluates FORMS with names rebound to temporary variables, ensuring
that each is evaluated only once.
\end{macro}

\begin{function}{parse-body}{body \key documentation whole}{alexandria}{}
  Parses BODY into (values remaining-forms declarations doc-string).
Documentation strings are recognized only if DOCUMENTATION is true.
Syntax errors in body are signalled and WHOLE is used in the signal
arguments when given.
\end{function}

\begin{function}{parse-ordinary-lambda-list}{lambda-list}{alexandria}{}
  Parses an ordinary lambda-list, returning as multiple values:

 1. Required parameters.
 2. Optional parameter specifications, normalized into form (NAME INIT SUPPLIEDP)
    where SUPPLIEDP is NIL if not present.
 3. Name of the rest parameter, or NIL.
 4. Keyword parameter specifications, normalized into form ((KEYWORD-NAME NAME) INIT SUPPLIEDP)
    where SUPPLIEDP is NIL if not present.
 5. Boolean indicating \&ALLOW-OTHER-KEYS presence.
 6. \&AUX parameter specifications, normalized into form (NAME INIT).

Signals a PROGRAM-ERROR is the lambda-list is malformed.
\end{function}

\begin{macro}{rebind}{bindings \body body}{arnesi}{}
  
\end{macro}

\begin{macro}{rebinding}{bindings \body body}{arnesi}{}
  Bind each var in BINDINGS to a gensym, bind the gensym to
var's value via a let, return BODY's value wrapped in this let.

Evaluates a series of forms in the lexical environment that is
formed by adding the binding of each VAR to a fresh, uninterned
symbol, and the binding of that fresh, uninterned symbol to VAR's
original value, i.e., its value in the current lexical
environment.

The uninterned symbol is created as if by a call to GENSYM with the
string denoted by PREFIX - or, if PREFIX is not supplied, the string
denoted by VAR - as argument.

The forms are evaluated in order, and the values of all but the last
are discarded (that is, the body is an implicit PROGN).
\end{macro}

