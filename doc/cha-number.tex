\chapter{Numbers}

\section{Predicates}
\label{sec:number-predicates}

\begin{function}{complexp}{object}
  Return true if OBJECT is a COMPLEX, and NIL otherwise.
\end{function}

\begin{function}{evenp}{number}
  Is this integer even?
\end{function}

\begin{function}{floatp}{object}
  Return true if OBJECT is a FLOAT, and NIL otherwise.
\end{function}

\begin{function}{integerp}{object}
  Return true if OBJECT is an INTEGER, and NIL otherwise.
\end{function}

\begin{function}{logbitp}{index integer}
  Predicate returns T if bit index of integer is a 1.
\end{function}

\begin{function}{minusp}{number}
  Is this real number strictly negative?
\end{function}

\begin{function}{numberp}{object}
  Return true if OBJECT is a NUMBER, and NIL otherwise.
\end{function}

\begin{function}{oddp}{number}
  Is this integer odd?
\end{function}

\begin{function}{plusp}{number}
  Is this real number strictly positive?
\end{function}

\begin{function}{random-state-p}{object}
  Returns true if \var{object} is of type \typer{random-state},
  otherwise returns NIL.
\end{function}

\begin{function}{rationalp}{object}
  Return true if OBJECT is a RATIONAL, and NIL otherwise.
\end{function}

\begin{function}{realp}{object}
  Return true if OBJECT is a REAL, and NIL otherwise.
\end{function}

\begin{function}{zerop}{number}
  Is this number zero?
\end{function}

The following functions are predicates for the sub-interval numerical
types described in CDR 5
\url{http://cdr.eurolisp.org/document/5/extra-num-types.html}.

\begin{function}{negative-rational-p}{n}[cl-ext]
  
\end{function}

\begin{function}{non-negative-double-float-p}{n}[cl-ext]
  
\end{function}

\begin{function}{negative-fixnum-p}{n}[cl-ext]
  
\end{function}

\begin{function}{negative-double-float-p}{n}[cl-ext]
  
\end{function}

\begin{function}{negative-float-p}{n}[cl-ext]
  
\end{function}

\begin{function}{negative-integer-p}{n}[cl-ext]
  
\end{function}

\begin{function}{negative-long-float-p}{n}[cl-ext]
  
\end{function}

\begin{function}{negative-real-p}{n}[cl-ext]
  
\end{function}

\begin{function}{negative-short-float-p}{n}[cl-ext]
  
\end{function}

\begin{function}{negative-single-float-p}{n}[cl-ext]
  
\end{function}

\begin{function}{non-negative-fixnum-p}{n}[cl-ext]
  
\end{function}

\begin{function}{non-negative-float-p}{n}[cl-ext]
  
\end{function}

\begin{function}{non-negative-integer-p}{n}[cl-ext]
  
\end{function}

\begin{function}{non-negative-long-float-p}{n}[cl-ext]
  
\end{function}

\begin{function}{non-negative-rational-p}{n}[cl-ext]
  
\end{function}

\begin{function}{non-negative-real-p}{n}[cl-ext]
  
\end{function}

\begin{function}{non-negative-short-float-p}{n}[cl-ext]
  
\end{function}

\begin{function}{non-negative-single-float-p}{n}[cl-ext]
  
\end{function}

\begin{function}{non-positive-double-float-p}{n}[cl-ext]
  
\end{function}

\begin{function}{non-positive-fixnum-p}{n}[cl-ext]
  
\end{function}

\begin{function}{non-positive-float-p}{n}[cl-ext]
  
\end{function}

\begin{function}{non-positive-integer-p}{n}[cl-ext]
  
\end{function}

\begin{function}{non-positive-long-float-p}{n}[cl-ext]
  
\end{function}

\begin{function}{non-positive-rational-p}{n}[cl-ext]
  
\end{function}

\begin{function}{non-positive-real-p}{n}[cl-ext]
  
\end{function}

\begin{function}{non-positive-short-float-p}{n}[cl-ext]
  
\end{function}

\begin{function}{non-positive-single-float-p}{n}[cl-ext]
  
\end{function}

\begin{function}{positive-double-float-p}{n}[cl-ext]
  
\end{function}

\begin{function}{positive-fixnum-p}{n}[cl-ext]
  
\end{function}

\begin{function}{positive-float-p}{n}[cl-ext]
  
\end{function}

\begin{function}{positive-integer-p}{n}[cl-ext]
  
\end{function}

\begin{function}{positive-long-float-p}{n}[cl-ext]
  
\end{function}

\begin{function}{positive-rational-p}{n}[cl-ext]
  
\end{function}

\begin{function}{positive-real-p}{n}[cl-ext]
  
\end{function}

\begin{function}{positive-short-float-p}{n}[cl-ext]
  
\end{function}

\begin{function}{positive-single-float-p}{n}[cl-ext]
  
\end{function}

\section{Number comparison}
\label{sec:number-comparison}

\begin{function}{/=}{number \rest more-numbers}
  Return T if no two of its arguments are numerically equal, NIL otherwise.
\end{function}

\begin{function}{=}{number \rest more-numbers}
  Return T if all of its arguments are numerically equal, NIL otherwise.
\end{function}

\begin{function}{>}{number \rest more-numbers}
  Return T if its arguments are in strictly decreasing order, NIL otherwise.
\end{function}

\begin{function}{>=}{number \rest more-numbers}
  Return T if arguments are in strictly non-increasing order, NIL otherwise.
\end{function}

\begin{function}{<}{number \rest more-numbers}
  Return T if its arguments are in strictly increasing order, NIL otherwise.
\end{function}

\begin{function}{<=}{number \rest more-numbers}
  Return T if arguments are in strictly non-decreasing order, NIL otherwise.
\end{function}

\begin{function}{max}{number \rest more-numbers}
  Return the greatest of its arguments; among EQUALP greatest, return
the first.
\end{function}

\begin{function}{min}{number \rest more-numbers}
  Return the least of its arguments; among EQUALP least, return
the first.
\end{function}

\begin{macro}{maxf}{place \rest numbers \env env}[cl-ext]
  Modify-macro for MAX. Sets place designated by the first argument to the
maximum of its original value and NUMBERS.
\end{macro}

\begin{macro}{minf}{place \rest numbers \env env}[cl-ext]
  Modify-macro for MIN. Sets place designated by the first argument to the
minimum of its original value and NUMBERS.
\end{macro}

\section{Arithmetic Operations}
\label{sec:arithm-oper}

\begin{function}{*}{\rest args}
  Return the product of its arguments. With no args, returns 1.
\end{function}

\begin{function}{+}{\rest args}
  Return the sum of its arguments. With no args, returns 0.
\end{function}

\begin{function}{-}{number \rest more-numbers}
  Subtract the second and all subsequent arguments from the first;
  or with one argument, negate the first argument.
\end{function}

\begin{function}{/}{number \rest more-numbers}
  Divide the first argument by each of the following arguments, in turn.
  With one argument, return reciprocal.
\end{function}

\begin{function}{1+}{number}
  Return NUMBER + 1.
\end{function}

\begin{function}{1-}{number}
  Return NUMBER - 1.
\end{function}

\begin{macro}{decf}{place \op delta \env env}
  The first argument is some location holding a number. This number is
  decremented by the second argument, DELTA, which defaults to 1.
\end{macro}

\begin{macro}{incf}{place \op delta \env env}
  The first argument is some location holding a number. This number is
  incremented by the second argument, DELTA, which defaults to 1.
\end{macro}

\begin{macro}{mulf}{place b \env env}[cl-ext]
  SETF NUM to the result of (* NUM B).
\end{macro}

\begin{macro}{divf}{place b \env env}[cl-ext]
  SETF NUM to the result of (/ NUM B).
\end{macro}

\begin{function}{abs}{number}
  Return the absolute value of the number.
\end{function}

\begin{function}{conjugate}{number}
  Return the complex conjugate of NUMBER. For non-complex numbers, this is
  an identity.
\end{function}

\begin{function}{gcd}{\rest integers}
  Return the greatest common divisor of the arguments, which must be
  integers. Gcd with no arguments is defined to be 0.
\end{function}

\begin{function}{lcm}{\rest integers}
  Return the least common multiple of one or more integers. LCM of no
  arguments is defined to be 1.
\end{function}

\begin{function}{signum}{number}
  Return the signal of a number as -1, 0, or 1.
\end{function}

\begin{function}{clamp}{number min max}[cl-ext]
  Clamps the NUMBER into [MIN, MAX] range. Returns MIN if NUMBER
  lesser then MIN and MAX if NUMBER is greater then MAX, otherwise
  returns NUMBER. This is a convenient way of limiting values to a set
  boundary.
\end{function}

%%% FIXME, ! as function
\begin{function}{factorial}{n}[cl-ext]
  Factorial of non-negative integer N. The function ! is an
  alternative name.
\end{function}

\begin{function}{subfactorial}{n}[cl-ext]
  Subfactorial of the non-negative integer N.
\end{function}

\begin{function}{lerp}{v a b}[cl-ext]
  Returns the result of linear interpolation between A and B, using the
interpolation coefficient V.
\end{function}

\begin{function}{mean}{sample}[cl-ext]
  Returns the mean of SAMPLE. SAMPLE must be a sequence of numbers.
\end{function}

\begin{function}{median}{sample}[cl-ext]
  Returns median of SAMPLE. SAMPLE must be a sequence of real numbers.
\end{function}

\begin{function}{variance}{sample \key biased}[cl-ext]
  Variance of SAMPLE. Returns the biased variance if BIASED is true (the default),
and the unbiased estimator of variance if BIASED is false. SAMPLE must be a
sequence of numbers.
\end{function}

\section{Exponential and Logarithmic Functions}
\label{sec:expon-logar-funct}

\begin{function}[10caret]{10\^{}}{n}[cl-ext]
  Return 10 to the \var{n}.
\end{function}

\begin{function}{exp}{number}
  Return e raised to the power NUMBER.
\end{function}

\begin{function}{expt}{base power}
  Return BASE raised to the POWER.
\end{function}

\begin{function}{log}{number \op base}
  Return the logarithm of NUMBER in the base BASE, which defaults to e.
\end{function}

\begin{function}{isqrt}{n}
  Return the root of the nearest integer less than n which is a perfect
   square.
\end{function}

\begin{function}{sqrt}{number}
  Return the square root of NUMBER.
\end{function}

\section{Trigonometric functions}
\label{sec:trig-funct}

See also \funr{abs} and \funr{signum} since they are extended to operate
on complex numbers.

\begin{constant}{pi}{}
  
\end{constant}

\begin{function}{acos}{number}
  Return the arc cosine of NUMBER.
\end{function}

\begin{function}{acosh}{number}
  Return the hyperbolic arc cosine of NUMBER.
\end{function}

\begin{function}{asin}{number}
  Return the arc sine of NUMBER.
\end{function}

\begin{function}{asinh}{number}
  Return the hyperbolic arc sine of NUMBER.
\end{function}

\begin{function}{atan}{y \op x}
  Return the arc tangent of Y if X is omitted or Y/X if X is supplied.
\end{function}

\begin{function}{atanh}{number}
  Return the hyperbolic arc tangent of NUMBER.
\end{function}

\begin{function}{cis}{theta}
  Return cos(Theta) + i sin(Theta), i.e. exp(i Theta).
\end{function}

\begin{function}{cos}{number}
  Return the cosine of NUMBER.
\end{function}

\begin{function}{cosh}{number}
  Return the hyperbolic cosine of NUMBER.
\end{function}

\begin{function}{phase}{number}
  Return the angle part of the polar representation of a complex number.
  For complex numbers, this is (atan (imagpart number) (realpart number)).
  For non-complex positive numbers, this is 0. For non-complex negative
  numbers this is PI.
\end{function}

\begin{function}{sin}{number}
  Return the sine of NUMBER.
\end{function}

\begin{function}{sinh}{number}
  Return the hyperbolic sine of NUMBER.
\end{function}

\begin{function}{tan}{number}
  Return the tangent of NUMBER.
\end{function}

\begin{function}{tanh}{number}
  Return the hyperbolic tangent of NUMBER.
\end{function}

\section{Type Conversions and Component Extractions}
\label{sec:type-conv-comp}

\begin{function}{float}{number \op other}
  Converts any REAL to a float. If OTHER is not provided, it returns a
  SINGLE-FLOAT if NUMBER is not already a FLOAT. If OTHER is provided, the
  result is the same float format as OTHER.
\end{function}

\begin{function}{denominator}{number}
  Return the denominator of NUMBER, which must be rational.
\end{function}

\begin{function}{numerator}{number}
  Return the numerator of NUMBER, which must be rational.
\end{function}

\begin{function}{rational}{x}
  RATIONAL produces a rational number for any real numeric argument. This is
  more efficient than RATIONALIZE, but it assumes that floating-point is
  completely accurate, giving a result that isn't as pretty.
\end{function}

\begin{function}{rationalize}{x}
  Converts any REAL to a RATIONAL.  Floats are converted to a simple rational
  representation exploiting the assumption that floats are only accurate to
  their precision.  RATIONALIZE (and also RATIONAL) preserve the invariant:
      (= x (float (rationalize x) x))
\end{function}

\begin{function}{mod}{number divisor}
  Return second result of FLOOR.
\end{function}

\begin{function}{rem}{number divisor}
  Return second result of TRUNCATE.
\end{function}

\begin{function}{ceiling}{number \op divisor}
  Return the smallest integer not less than number, or number/divisor.
  The second returned value is the remainder.
\end{function}

\begin{function}{floor}{number \op divisor}
  Return the greatest integer not greater than number, or number/divisor.
  The second returned value is (mod number divisor).
\end{function}

\begin{function}{round}{number \op divisor}
  Rounds number (or number/divisor) to nearest integer.
  The second returned value is the remainder.
\end{function}

\begin{function}{truncate}{number \op divisor}
  Return number (or number/divisor) as an integer, rounded toward 0.
  The second returned value is the remainder.
\end{function}

\begin{function}{fceiling}{number \op divisor}
  Same as CEILING, but returns first value as a float.
\end{function}

\begin{function}{ffloor}{number \op divisor}
  Same as FLOOR, but returns first value as a float.
\end{function}

\begin{function}{fround}{number \op divisor}
  Same as ROUND, but returns first value as a float.
\end{function}

\begin{function}{ftruncate}{number \op divisor}
  Same as TRUNCATE, but returns first value as a float.
\end{function}

\begin{function}{decode-float}{f}
  Return three values:
   1) a floating-point number representing the significand. This is always
      between 0.5 (inclusive) and 1.0 (exclusive).
   2) an integer representing the exponent.
   3) -1.0 or 1.0 (i.e. the sign of the argument.)
\end{function}

\begin{function}{scale-float}{f ex}
  Return the value (* f (expt (float 2 f) ex)), but with no unnecessary loss
  of precision or overflow.
\end{function}

\begin{function}{float-digits}{f}
  
\end{function}

\begin{function}{float-precision}{f}
  Return a non-negative number of significant digits in its float argument.
  Will be less than FLOAT-DIGITS if denormalized or zero.
\end{function}

\begin{function}{float-radix}{x}
  Return (as an integer) the radix b of its floating-point argument.
\end{function}

\begin{function}{float-sign}{float1 \op float2}
  Return a floating-point number that has the same sign as
   FLOAT1 and, if FLOAT2 is given, has the same absolute value
   as FLOAT2.
\end{function}

\begin{function}{integer-decode-float}{x}
  Return three values:
   1) an integer representation of the significand.
   2) the exponent for the power of 2 that the significand must be multiplied
      by to get the actual value. This differs from the DECODE-FLOAT exponent
      by FLOAT-DIGITS, since the significand has been scaled to have all its
      digits before the radix point.
   3) -1 or 1 (i.e. the sign of the argument.)
\end{function}

\begin{function}{complex}{realpart \op imagpart}
  Return a complex number with the specified real and imaginary components.
\end{function}

\begin{function}{realpart}{number}
  Extract the real part of a number.
\end{function}

\begin{function}{imagpart}{number}
  Extract the imaginary part of a number.
\end{function}

\section{Logic operations}
\label{sec:logic-operations}

\begin{function}{boole}{op integer1 integer2}
  Bit-wise boolean function on two integers. Function chosen by OP:
        0       BOOLE-CLR
        1       BOOLE-SET
        2       BOOLE-1
        3       BOOLE-2
        4       BOOLE-C1
        5       BOOLE-C2
        6       BOOLE-AND
        7       BOOLE-IOR
        8       BOOLE-XOR
        9       BOOLE-EQV
        10      BOOLE-NAND
        11      BOOLE-NOR
        12      BOOLE-ANDC1
        13      BOOLE-ANDC2
        14      BOOLE-ORC1
        15      BOOLE-ORC2
\end{function}

\begin{constant}{boole-1}{}
  
\end{constant}

\begin{constant}{boole-2}{}
  
\end{constant}

\begin{constant}{boole-and}{}
  
\end{constant}

\begin{constant}{boole-andc1}{}
  
\end{constant}

\begin{constant}{boole-andc2}{}
  
\end{constant}

\begin{constant}{boole-c1}{}
  
\end{constant}

\begin{constant}{boole-c2}{}
  
\end{constant}

\begin{constant}{boole-clr}{}
  
\end{constant}

\begin{constant}{boole-eqv}{}
  
\end{constant}

\begin{constant}{boole-ior}{}
  
\end{constant}

\begin{constant}{boole-nand}{}
  
\end{constant}

\begin{constant}{boole-nor}{}
  
\end{constant}

\begin{constant}{boole-orc1}{}
  
\end{constant}

\begin{constant}{boole-orc2}{}
  
\end{constant}

\begin{constant}{boole-set}{}
  
\end{constant}

\begin{constant}{boole-xor}{}
  
\end{constant}

\begin{function}{logand}{\rest integers}
  Return the bit-wise and of its arguments. Args must be integers.
\end{function}

\begin{function}{logandc1}{integer1 integer2}
  Bitwise AND (LOGNOT INTEGER1) with INTEGER2.
\end{function}

\begin{function}{logandc2}{integer1 integer2}
  Bitwise AND INTEGER1 with (LOGNOT INTEGER2).
\end{function}

\begin{function}{logcount}{integer}
  Count the number of 1 bits if INTEGER is positive, and the number of 0 bits
  if INTEGER is negative.
\end{function}

\begin{function}{logeqv}{\rest integers}
  Return the bit-wise equivalence of its arguments. Args must be integers.
\end{function}

\begin{function}{logior}{\rest integers}
  Return the bit-wise or of its arguments. Args must be integers.
\end{function}

\begin{function}{lognand}{integer1 integer2}
  Complement the logical AND of INTEGER1 and INTEGER2.
\end{function}

\begin{function}{lognor}{integer1 integer2}
  Complement the logical AND of INTEGER1 and INTEGER2.
\end{function}

\begin{function}{lognot}{number}
  Return the bit-wise logical not of integer.
\end{function}

\begin{function}{logorc1}{integer1 integer2}
  Bitwise OR (LOGNOT INTEGER1) with INTEGER2.
\end{function}

\begin{function}{logorc2}{integer1 integer2}
  Bitwise OR INTEGER1 with (LOGNOT INTEGER2).
\end{function}

\begin{function}{logtest}{integer1 integer2}
  Predicate which returns T if logand of integer1 and integer2 is not zero.
\end{function}

\begin{function}{logxor}{\rest integers}
  Return the bit-wise exclusive or of its arguments. Args must be integers.
\end{function}

\begin{function}{ash}{integer count}
  Shifts integer left by count places preserving sign. - count shifts right.
\end{function}

\begin{function}{integer-length}{integer}
  Return the number of non-sign bits in the twos-complement representation
  of INTEGER.
\end{function}

\section{Byte Manipulation Functions}
\label{sec:byte-manip-funct}

\begin{function}{byte}{size position}
  Return a byte specifier which may be used by other byte functions
  (e.g. LDB).
\end{function}

\begin{function}{byte-position}{bytespec}
  Return the position part of the byte specifier bytespec.
\end{function}

\begin{function}{byte-size}{bytespec}
  Return the size part of the byte specifier bytespec.
\end{function}

\begin{function}{deposit-field}{newbyte bytespec integer}
  Return new integer with newbyte in specified position, newbyte is not right justified.
\end{function}

\begin{function}{dpb}{newbyte bytespec integer}
  Return new integer with newbyte in specified position, newbyte is right justified.
\end{function}

\begin{function}{ldb-test}{bytespec integer}
  Return T if any of the specified bits in integer are 1's.
\end{function}

\begin{accessor}{ldb}{bytespec integer}
  Extract the specified byte from integer, and right justify result.
\end{accessor}

\begin{accessor}{mask-field}{bytespec integer}
  Extract the specified byte from integer,  but do not right justify result.
\end{accessor}

\section{Random numbers}
\label{sec:random-numbers}

See also \funr{random-state-p}

\begin{function}{random}{arg \op state}
  
\end{function}

\begin{function}{make-random-state}{\op state}
  Make a random state object. If STATE is not supplied, return a copy
  of the default random state. If STATE is a random state, then return a
  copy of it. If STATE is T then return a random state generated from
  the universal time.
\end{function}

\begin{class}{random-state}{}
  
\end{class}

\begin{variable}{*random-state*}{}
  
\end{variable}

\section{Implementation Parameters}
\label{sec:impl-param}

\begin{constant}{most-negative-double-float}{}
  
\end{constant}

\begin{constant}{most-negative-fixnum}{}
  
\end{constant}

\begin{constant}{most-negative-long-float}{}
  
\end{constant}

\begin{constant}{most-negative-short-float}{}
  
\end{constant}

\begin{constant}{most-negative-single-float}{}
  
\end{constant}

\begin{constant}{most-positive-double-float}{}
  
\end{constant}

\begin{constant}{most-positive-fixnum}{}
  
\end{constant}

\begin{constant}{most-positive-long-float}{}
  
\end{constant}

\begin{constant}{most-positive-short-float}{}
  
\end{constant}

\begin{constant}{most-positive-single-float}{}
  
\end{constant}

\begin{constant}{least-negative-double-float}{}
  
\end{constant}

\begin{constant}{least-negative-long-float}{}
  
\end{constant}

\begin{constant}{least-negative-normalized-double-float}{}
  
\end{constant}

\begin{constant}{least-negative-normalized-long-float}{}
  
\end{constant}

\begin{constant}{least-negative-normalized-short-float}{}
  
\end{constant}

\begin{constant}{least-negative-normalized-single-float}{}
  
\end{constant}

\begin{constant}{least-negative-short-float}{}
  
\end{constant}

\begin{constant}{least-negative-single-float}{}
  
\end{constant}

\begin{constant}{least-positive-double-float}{}
  
\end{constant}

\begin{constant}{least-positive-long-float}{}
  
\end{constant}

\begin{constant}{least-positive-normalized-double-float}{}
  
\end{constant}

\begin{constant}{least-positive-normalized-long-float}{}
  
\end{constant}

\begin{constant}{least-positive-normalized-short-float}{}
  
\end{constant}

\begin{constant}{least-positive-normalized-single-float}{}
  
\end{constant}

\begin{constant}{least-positive-short-float}{}
  
\end{constant}

\begin{constant}{least-positive-single-float}{}
  
\end{constant}

\begin{constant}{double-float-epsilon}{}
  
\end{constant}

\begin{constant}{double-float-negative-epsilon}{}
  
\end{constant}

\begin{constant}{long-float-epsilon}{}
  
\end{constant}

\begin{constant}{long-float-negative-epsilon}{}
  
\end{constant}

\begin{constant}{short-float-epsilon}{}
  
\end{constant}

\begin{constant}{short-float-negative-epsilon}{}
  
\end{constant}

\begin{constant}{single-float-epsilon}{}
  
\end{constant}

\begin{constant}{single-float-negative-epsilon}{}
  
\end{constant}

\section{Other extensions}
\label{sec:other-extensions}

\begin{function}{gaussian-random}{\op min max}[cl-ext]
  Returns two gaussian random double floats as the primary and secondary value,
optionally constrained by MIN and MAX. Gaussian random numbers form a standard
normal distribution around 0.0d0.
\end{function}

\begin{function}{binomial-coefficient}{n k}[cl-ext]
  Binomial coefficient of N and K, also expressed as N choose K. This is the
number of K element combinations given N choises. N must be equal to or
greater then K.
\end{function}

\begin{function}{count-permutations}{n \op k}[cl-ext]
  Number of K element permutations for a sequence of N objects.
R defaults to N
\end{function}

\begin{function}{standard-deviation}{sample \key biased}[cl-ext]
  Standard deviation of SAMPLE. Returns the biased standard deviation if
BIASED is true (the default), and the square root of the unbiased estimator
for variance if BIASED is false (which is not the same as the unbiased
estimator for standard deviation). SAMPLE must be a sequence of numbers.
\end{function}

%%% Local Variables:
%%% mode: latex
%%% TeX-master: cl-dist-manual.tex
%%% End:
