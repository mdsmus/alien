\documentclass[10pt,english]{book}
\usepackage{lisp}

\makeindex

\title{The Common Lisp Manual}
\author{Pedro Kroger, editor}
\date{May 19, 2009}
\release{1.0}

\begin{document}

\frontmatter

\maketitle
\tableofcontents

\mainmatter

\part{The Common Lisp standard}
\label{part:common-lisp-standard}


\chapter*{Acknowledgement}
\label{cha:acknowledgement}


%%% Local Variables:
%%% mode: latex
%%% TeX-master: cl-dist-manual.tex
%%% End:

\chapter{Introduction}

%%% Local Variables:
%%% mode: latex
%%% TeX-master: cl-dist-manual.tex
%%% End:

\input{cha-number}
\input{cha-character}
\chapter{Strings}
\label{cha:strings}

See also the chapter on sequences () for functions to split strings and
since strings are array see also the chapter on arrays.

\begin{type}{string-designator}{}[cl-ext]
  A string designator type. A string designator is either a string, a
  symbol, or a character.
\end{type}

\section{Predicates}
\label{sec:string-predicates}

See also string-greaterp, \funr{string-lessp},
\funr{string-not-greaterp}, and \funr{string-not-lessp}.

\begin{function}{stringp}{object}
  Return true if OBJECT is a STRING, and NIL otherwise.
\end{function}

\begin{function}{simple-string-p}{object}
  Return true if OBJECT is a SIMPLE-STRING, and NIL otherwise.
\end{function}

\begin{function}{empty-string-p}{string}[cl-ext]
  Indicates, if a given string is empty (or being nil)
\end{function}

\section{String constants}
\label{sec:string-constants}

\begin{constant}{+lower-case-ascii-alphabet+}{}[cl-ext]
  All the lower case letters in 7 bit ASCII.
\end{constant}

\begin{constant}{+upper-case-ascii-alphabet+}{}[cl-ext]
  All the upper case letters in 7 bit ASCII.
\end{constant}

\begin{constant}{+alphanumeric-ascii-alphabet+}{}[cl-ext]
  All the letters and numbers in 7 bit ASCII.
\end{constant}

\begin{constant}{+ascii-alphabet+}{}[cl-ext]
  All letters in 7 bit ASCII.
\end{constant}

\begin{constant}{+base64-alphabet+}{}[cl-ext]
  All the characters allowed in base64 encoding.
\end{constant}

\section{String access}
\label{sec:string-access}

\begin{accessor}{char}{string index}
  Given a string and a non-negative integer index less than the length of
  the string, returns the character object representing the character at
  that position in the string.
\end{accessor}

\begin{accessor}{schar}{string index}
  SCHAR returns the character object at an indexed position in a string
   just as CHAR does, except the string must be a simple-string.
\end{accessor}

\section{Comparison}
\label{sec:string-comparison}

\begin{function}{string/=}{string1 string2 \key start1 end1 start2 end2}
  Given two strings, if the first string is not lexicographically equal
  to the second string, returns the longest common prefix (using char=)
  of the two strings. Otherwise, returns ().
\end{function}

\begin{function}{string<}{string1 string2 \key start1 end1 start2 end2}
  Given two strings, if the first string is lexicographically less than
  the second string, returns the longest common prefix (using char=)
  of the two strings. Otherwise, returns ().
\end{function}

\begin{function}{string<=}{string1 string2 \key start1 end1 start2 end2}
  Given two strings, if the first string is lexicographically less than
  or equal to the second string, returns the longest common prefix
  (using char=) of the two strings. Otherwise, returns ().
\end{function}

\begin{function}{string=}{string1 string2 \key start1 end1 start2 end2}
  Given two strings (string1 and string2), and optional integers start1,
  start2, end1 and end2, compares characters in string1 to characters in
  string2 (using char=).
\end{function}

\begin{function}{string>}{string1 string2 \key start1 end1 start2 end2}
  Given two strings, if the first string is lexicographically greater than
  the second string, returns the longest common prefix (using char=)
  of the two strings. Otherwise, returns ().
\end{function}

\begin{function}{string>=}{string1 string2 \key start1 end1 start2 end2}
  Given two strings, if the first string is lexicographically greater
  than or equal to the second string, returns the longest common prefix
  (using char=) of the two strings. Otherwise, returns ().
\end{function}

The following functions are just like \fun{string<}, \fun{string>}, \fun{string<=},
\fun{string>=}, and \fun{string/=} except they are case insensitive.

\begin{function}{string-equal}{string1 string2 \key start1 end1 start2 end2}
  Given two strings (string1 and string2), and optional integers start1,
  start2, end1 and end2, compares characters in string1 to characters in
  string2 (using char-equal).
\end{function}

\begin{function}{string-greaterp}{string1 string2 \key start1 end1 start2 end2}
  Given two strings, if the first string is lexicographically greater than
  the second string, returns the longest common prefix (using char-equal)
  of the two strings. Otherwise, returns ().
\end{function}

\begin{function}{string-lessp}{string1 string2 \key start1 end1 start2 end2}
  Given two strings, if the first string is lexicographically less than
  the second string, returns the longest common prefix (using char-equal)
  of the two strings. Otherwise, returns ().
\end{function}

\begin{function}{string-not-equal}{string1 string2 \key start1 end1 start2 end2}
  Given two strings, if the first string is not lexicographically equal
  to the second string, returns the longest common prefix (using char-equal)
  of the two strings. Otherwise, returns ().
\end{function}

\begin{function}{string-not-greaterp}{string1 string2 \key start1 end1 start2 end2}
  Given two strings, if the first string is lexicographically less than
  or equal to the second string, returns the longest common prefix
  (using char-equal) of the two strings. Otherwise, returns ().
\end{function}

\begin{function}{string-not-lessp}{string1 string2 \key start1 end1 start2 end2}
  Given two strings, if the first string is lexicographically greater
  than or equal to the second string, returns the longest common prefix
  (using char-equal) of the two strings. Otherwise, returns ().
\end{function}

\section{Construction and manipulation}
\label{sec:constr-manip}

\begin{function}{string}{x}
  Coerces X into a string. If X is a string, X is returned. If X is a
   symbol, X's pname is returned. If X is a character then a one element
   string containing that character is returned. If X cannot be coerced
   into a string, an error occurs.
\end{function}

\begin{function}{make-string}{count \key element-type (initial-element fill-char)}
  Given a character count and an optional fill character, makes and returns a
new string COUNT long filled with the fill character.
\end{function}

\begin{function}{random-string}{\op length alphabet}[cl-ext]
  Returns a random alphabetic string. The returned string will contain
  LENGTH characters chosen from the vector ALPHABET.
\end{function}

\begin{function}{string-trim}{char-bag string}
  
\end{function}

\begin{function}{string-left-trim}{char-bag string}
  
\end{function}

\begin{function}{string-right-trim}{char-bag string}
  
\end{function}

\begin{function}{string-downcase}{string \key start end}
  
\end{function}

\begin{function}{string-upcase}{string \key start end}
  
\end{function}

\begin{function}{string-capitalize}{string \key start end}
  
\end{function}

\begin{function}{nstring-downcase}{string \key start end}
  
\end{function}

\begin{function}{nstring-upcase}{string \key start end}
  
\end{function}

\begin{function}{nstring-capitalize}{string \key start end}
  
\end{function}

\begin{function}{strcat}{\rest items}[cl-ext]
  Returns a fresh string consisting of ITEMS concat'd together.
\end{function}

\begin{function}{strcat*}{string-designators}[cl-ext]
  Concatenate all the strings in STRING-DESIGNATORS.
\end{function}

\begin{function}{join-strings}{strings}[cl-ext]
  Concatenate strings. It's a fast shorthand for \code{(concatenate
    'string string1 \ldots stringn)}.
\end{function}

\begin{function}{fold-strings}{}[cl-ext]
  
\end{function}

\begin{function}{split-tabs}{string}[cl-ext]
  Utility function to split a string by tabs and remove empty
  subsequences.
\end{function}

\begin{function}{split-newline}{string}[cl-ext]
  Utility function to split a string by newlines and remove empty
  subsequences.
\end{function}

\begin{function}{replace-all}{part string replacement \key test}[cl-ext]
  Returns a new string in which all the occurences of the part is
  replaced with replacement. It was taken from the cl-cookbook, it's
  not as optimized as cl-ppcre, but if you just want to replace
  strings it may be faster in some cases.
\end{function}

\begin{function}{string-from-array}{array \key start end}[cl-ext]
  Assuming ARRAY is an array of ASCII chars encoded as bytes return
the corresponding string. Respect the C convention of null terminating
strings. START and END specify the zero indexed offsets of a sub range
of ARRAY.
\end{function}

\begin{function}{octets-to-string}{octets encoding}[cl-ext]
  
\end{function}

\begin{function}{string-to-octets}{string encoding}[cl-ext]
  Convert STRING, a list string, a vector of bytes according to ENCODING.

ENCODING is a keyword representing the desired character
encoding. We gurantee that :UTF-8, :UTF-16 and :ISO-8859-1 will
work as expected. Any other values are simply passed to the
underlying lisp's function and the results are implementation
dependant.

On CLISP we intern the ENCODING symbol in the CHARSET package and
pass that. On SBCL we simply pass the keyword.
\end{function}

\begin{function}{encoding-keyword-to-native}{encoding}[cl-ext]
  Convert ENCODING, a keyword, to an object the native list
accepts as an encoding.

ENCODING can be: :UTF-8, :UTF-16, or :US-ASCII and specify the
corresponding encodings. Any other keyword is passed, as is, to
the underlying lisp.
\end{function}

\section{String reading}
\label{sec:string-reading}

\begin{function}{parse-integer}{string \key start end radix junk-allowed}
  Examine the substring of string delimited by start and end
  (default to the beginning and end of the string)  It skips over
  whitespace characters and then tries to parse an integer. The
  radix parameter must be between 2 and 36.
\end{function}

\begin{function}{parse-float}{float-string \key start end radix junk-allowed type decimal-character}[cl-ext]
\end{function}


%%% Local Variables:
%%% mode: latex
%%% TeX-master: cl-dist-manual.tex
%%% End:

\chapter{Arrays and vectors}
\label{cha:arrays}

\section{Predicates}
\label{sec:array-predicates}

\begin{function}{arrayp}{object}
  Return true if OBJECT is an ARRAY, and NIL otherwise.
\end{function}

\begin{function}{adjustable-array-p}{array}
  Return T if (ADJUST-ARRAY ARRAY...) would return an array identical
   to the argument, this happens for complex arrays.
\end{function}

\begin{function}{array-has-fill-pointer-p}{array}
  Return T if the given ARRAY has a fill pointer, or NIL otherwise.
\end{function}

\begin{function}{array-in-bounds-p}{array \rest subscripts}
  Return T if the SUBSCIPTS are in bounds for the ARRAY, NIL otherwise.
\end{function}

\begin{function}{vectorp}{object}
  Return true if OBJECT is a VECTOR, and NIL otherwise.
\end{function}

\begin{function}{bit-vector-p}{object}
  Return true if OBJECT is a BIT-VECTOR, and NIL otherwise.
\end{function}

\begin{function}{simple-bit-vector-p}{object}
  Return true if OBJECT is a SIMPLE-BIT-VECTOR, and NIL otherwise.
\end{function}

\begin{function}{simple-vector-p}{object}
  Return true if OBJECT is a SIMPLE-VECTOR, and NIL otherwise.
\end{function}

\section{Creation}
\label{sec:array-creation}

\begin{function}{make-array}{dimensions \key element-type initial-element initial-contents adjustable
 fill-pointer displaced-to displaced-index-offset}
  
\end{function}

\begin{function}{make-displaced-array}{array \op start end}[cl-ext]
  
\end{function}

\begin{function}{copy-array}{array \key element-type fill-pointer adjustable}[cl-ext]
  Returns an undisplaced copy of ARRAY, with same fill-pointer
and adjustability (if any) as the original, unless overridden by
the keyword arguments.
\end{function}

\begin{constant}{array-dimension-limit}{}
  
\end{constant}

\begin{constant}{array-rank-limit}{}
  
\end{constant}

\begin{constant}{array-total-size-limit}{}
  
\end{constant}

\begin{function}{vector}{\rest objects}
  Construct a SIMPLE-VECTOR from the given objects.
\end{function}

\section{Access}
\label{sec:array-access}

\begin{accessor}{aref}{array \rest subscripts}
  Return the element of the ARRAY specified by the SUBSCRIPTS.
\end{accessor}

\begin{accessor}{svref}{simple-vector index}
  Return the INDEX'th element of the given Simple-Vector.
\end{accessor}

\section{Array information}
\label{sec:array-information}

\begin{function}{array-dimension}{array axis-number}
  Return the length of dimension AXIS-NUMBER of ARRAY.
\end{function}

\begin{function}{array-dimensions}{array}
  Return a list whose elements are the dimensions of the array
\end{function}

\begin{function}{array-displacement}{array}
  Return the values of :DISPLACED-TO and :DISPLACED-INDEX-offset
   options to MAKE-ARRAY, or NIL and 0 if not a displaced array.
\end{function}

\begin{function}{array-element-type}{array}
  Return the type of the elements of the array
\end{function}

\begin{function}{array-rank}{array}
  Return the number of dimensions of ARRAY.
\end{function}

\begin{function}{array-total-size}{array}
  Return the total number of elements in the Array.
\end{function}

\begin{function}{array-row-major-index}{array \rest subscripts}
  
\end{function}

\begin{accessor}{row-major-aref}{array index}
  Return the element of array corressponding to the row-major index. This is
   SETF'able.
\end{accessor}

\section{Functions on Arrays of Bits}
\label{sec:funct-arrays-bits}

An array of bits is a specialized array whose elements are only 0 or
1.

\begin{accessor}{bit}{bit-array \rest subscripts}
  Return the bit from the BIT-ARRAY at the specified SUBSCRIPTS.
\end{accessor}

\begin{accessor}{sbit}{simple-bit-array \rest subscripts}
  Return the bit from SIMPLE-BIT-ARRAY at the specified SUBSCRIPTS.
\end{accessor}

\begin{function}{bit-and}{bit-array-1 bit-array-2 \op result-bit-array}
  Perform a bit-wise LOGAND on the elements of BIT-ARRAY-1 and BIT-ARRAY-2,
  putting the results in RESULT-BIT-ARRAY. If RESULT-BIT-ARRAY is T,
  BIT-ARRAY-1 is used. If RESULT-BIT-ARRAY is NIL or omitted, a new array is
  created. All the arrays must have the same rank and dimensions.
\end{function}

\begin{function}{bit-andc1}{bit-array-1 bit-array-2 \op result-bit-array}
  Perform a bit-wise LOGANDC1 on the elements of BIT-ARRAY-1 and BIT-ARRAY-2,
  putting the results in RESULT-BIT-ARRAY. If RESULT-BIT-ARRAY is T,
  BIT-ARRAY-1 is used. If RESULT-BIT-ARRAY is NIL or omitted, a new array is
  created. All the arrays must have the same rank and dimensions.
\end{function}

\begin{function}{bit-andc2}{bit-array-1 bit-array-2 \op result-bit-array}
  Perform a bit-wise LOGANDC2 on the elements of BIT-ARRAY-1 and BIT-ARRAY-2,
  putting the results in RESULT-BIT-ARRAY. If RESULT-BIT-ARRAY is T,
  BIT-ARRAY-1 is used. If RESULT-BIT-ARRAY is NIL or omitted, a new array is
  created. All the arrays must have the same rank and dimensions.
\end{function}

\begin{function}{bit-eqv}{bit-array-1 bit-array-2 \op result-bit-array}
  Perform a bit-wise LOGEQV on the elements of BIT-ARRAY-1 and BIT-ARRAY-2,
  putting the results in RESULT-BIT-ARRAY. If RESULT-BIT-ARRAY is T,
  BIT-ARRAY-1 is used. If RESULT-BIT-ARRAY is NIL or omitted, a new array is
  created. All the arrays must have the same rank and dimensions.
\end{function}

\begin{function}{bit-ior}{bit-array-1 bit-array-2 \op result-bit-array}
  Perform a bit-wise LOGIOR on the elements of BIT-ARRAY-1 and BIT-ARRAY-2,
  putting the results in RESULT-BIT-ARRAY. If RESULT-BIT-ARRAY is T,
  BIT-ARRAY-1 is used. If RESULT-BIT-ARRAY is NIL or omitted, a new array is
  created. All the arrays must have the same rank and dimensions.
\end{function}

\begin{function}{bit-nand}{bit-array-1 bit-array-2 \op result-bit-array}
  Perform a bit-wise LOGNAND on the elements of BIT-ARRAY-1 and BIT-ARRAY-2,
  putting the results in RESULT-BIT-ARRAY. If RESULT-BIT-ARRAY is T,
  BIT-ARRAY-1 is used. If RESULT-BIT-ARRAY is NIL or omitted, a new array is
  created. All the arrays must have the same rank and dimensions.
\end{function}

\begin{function}{bit-nor}{bit-array-1 bit-array-2 \op result-bit-array}
  Perform a bit-wise LOGNOR on the elements of BIT-ARRAY-1 and BIT-ARRAY-2,
  putting the results in RESULT-BIT-ARRAY. If RESULT-BIT-ARRAY is T,
  BIT-ARRAY-1 is used. If RESULT-BIT-ARRAY is NIL or omitted, a new array is
  created. All the arrays must have the same rank and dimensions.
\end{function}

\begin{function}{bit-not}{bit-array \op result-bit-array}
  Performs a bit-wise logical NOT on the elements of BIT-ARRAY,
  putting the results in RESULT-BIT-ARRAY. If RESULT-BIT-ARRAY is T,
  BIT-ARRAY is used. If RESULT-BIT-ARRAY is NIL or omitted, a new array is
  created. Both arrays must have the same rank and dimensions.
\end{function}

\begin{function}{bit-orc1}{bit-array-1 bit-array-2 \op result-bit-array}
  Perform a bit-wise LOGORC1 on the elements of BIT-ARRAY-1 and BIT-ARRAY-2,
  putting the results in RESULT-BIT-ARRAY. If RESULT-BIT-ARRAY is T,
  BIT-ARRAY-1 is used. If RESULT-BIT-ARRAY is NIL or omitted, a new array is
  created. All the arrays must have the same rank and dimensions.
\end{function}

\begin{function}{bit-orc2}{bit-array-1 bit-array-2 \op result-bit-array}
  Perform a bit-wise LOGORC2 on the elements of BIT-ARRAY-1 and BIT-ARRAY-2,
  putting the results in RESULT-BIT-ARRAY. If RESULT-BIT-ARRAY is T,
  BIT-ARRAY-1 is used. If RESULT-BIT-ARRAY is NIL or omitted, a new array is
  created. All the arrays must have the same rank and dimensions.
\end{function}

\begin{function}{bit-xor}{bit-array-1 bit-array-2 \op result-bit-array}
  Perform a bit-wise LOGXOR on the elements of BIT-ARRAY-1 and BIT-ARRAY-2,
  putting the results in RESULT-BIT-ARRAY. If RESULT-BIT-ARRAY is T,
  BIT-ARRAY-1 is used. If RESULT-BIT-ARRAY is NIL or omitted, a new array is
  created. All the arrays must have the same rank and dimensions.
\end{function}

\section{Fill pointer}
\label{sec:fill-pointer}

\begin{accessor}{fill-pointer}{vector}
  Return the FILL-POINTER of the given VECTOR.
\end{accessor}

\begin{function}{vector-pop}{array}
  Decrease the fill pointer by 1 and return the element pointed to by the
  new fill pointer.
\end{function}

\begin{function}{vector-push}{new-el array}
  Attempt to set the element of ARRAY designated by its fill pointer
   to NEW-EL, and increment the fill pointer by one. If the fill pointer is
   too large, NIL is returned, otherwise the index of the pushed element is
   returned.
\end{function}

\begin{function}{vector-push-extend}{new-element vector \op min-extension}
  
\end{function}

\section{Changing dimensions}
\label{sec:changing-dimensions}

\begin{function}{adjust-array}{array dimensions \key element-type initial-element initial-contents
 fill-pointer displaced-to displaced-index-offset}
  Adjust ARRAY's dimensions to the given DIMENSIONS and stuff.
\end{function}

\section{Type upgrading}
\label{sec:type-upgrading}

\begin{function}{upgraded-array-element-type}{spec \op environment}
  Return the element type that will actually be used to implement an array
   with the specifier :ELEMENT-TYPE Spec.
\end{function}


%%% Local Variables:
%%% mode: latex
%%% TeX-master: cl-dist-manual.tex
%%% End:

\input{cha-lists}
\input{cha-sequence}
\input{cha-hash-table}
\input{cha-structure}
\chapter{Input and Output}

\section{Reader}

\begin{accessor}{readtable-case}{readtable}
  
\end{accessor}

\begin{function}{copy-readtable}{\op from-readtable to-readtable}
  
\end{function}

\begin{function}{get-dispatch-macro-character}{disp-char sub-char \op rt-designator}
  Return the macro character function for SUB-CHAR under DISP-CHAR
   or NIL if there is no associated function.
\end{function}

\begin{function}{get-macro-character}{char \op rt-designator}
  Return the function associated with the specified CHAR which is a macro
  character, or NIL if there is no such function. As a second value, return
  T if CHAR is a macro character which is non-terminating, i.e. which can
  be embedded in a symbol name.
\end{function}

\begin{function}{make-dispatch-macro-character}{char \op non-terminating-p rt}
  Cause CHAR to become a dispatching macro character in readtable (which
   defaults to the current readtable). If NON-TERMINATING-P, the char will
   be non-terminating.
\end{function}

\begin{function}{read}{\op stream eof-error-p eof-value recursive-p}
  Read the next Lisp value from STREAM, and return it.
\end{function}

\begin{function}{read-delimited-list}{endchar \op input-stream recursive-p}
  Read Lisp values from INPUT-STREAM until the next character after a
   value's representation is ENDCHAR, and return the objects as a list.
\end{function}

\begin{function}{read-from-string}{string \op eof-error-p eof-value \key start end preserve-whitespace}
  The characters of string are successively given to the lisp reader
   and the lisp object built by the reader is returned. Macro chars
   will take effect.
\end{function}

\begin{function}{read-preserving-whitespace}{\op stream eof-error-p eof-value recursive-p}
  Read from STREAM and return the value read, preserving any whitespace
   that followed the object.
\end{function}

\begin{function}{readtablep}{object}
  
\end{function}

\begin{function}{set-dispatch-macro-character}{disp-char sub-char function \op rt-designator}
  Cause FUNCTION to be called whenever the reader reads DISP-CHAR
   followed by SUB-CHAR.
\end{function}

\begin{function}{set-macro-character}{char function \op non-terminatingp rt-designator}
  Causes CHAR to be a macro character which invokes FUNCTION when seen
   by the reader. The NON-TERMINATINGP flag can be used to make the macro
   character non-terminating, i.e. embeddable in a symbol name.
\end{function}

\begin{function}{set-syntax-from-char}{to-char from-char \op to-readtable from-readtable}
  Causes the syntax of TO-CHAR to be the same as FROM-CHAR in the optional
readtable (defaults to the current readtable). The FROM-TABLE defaults to the
standard Lisp readtable when NIL.
\end{function}

\begin{macro}{with-standard-io-syntax}{\body body}
  Bind the reader and printer control variables to values that enable READ
   to reliably read the results of PRINT. These values are:
       *PACKAGE*                        the COMMON-LISP-USER package
       *PRINT-ARRAY*                    T
       *PRINT-BASE*                     10
       *PRINT-CASE*                     :UPCASE
       *PRINT-CIRCLE*                   NIL
       *PRINT-ESCAPE*                   T
       *PRINT-GENSYM*                   T
       *PRINT-LENGTH*                   NIL
       *PRINT-LEVEL*                    NIL
       *PRINT-LINES*                    NIL
       *PRINT-MISER-WIDTH*              NIL
       *PRINT-PRETTY*                   NIL
       *PRINT-RADIX*                    NIL
       *PRINT-READABLY*                 T
       *PRINT-RIGHT-MARGIN*             NIL
       *READ-BASE*                      10
       *READ-DEFAULT-FLOAT-FORMAT*      SINGLE-FLOAT
       *READ-EVAL*                      T
       *READ-SUPPRESS*                  NIL
       *READTABLE*                      the standard readtable
\end{macro}

\begin{class}{readtable}{}
  
\end{class}

\begin{variable}{*read-base*}{}
  
\end{variable}

\begin{variable}{*read-default-float-format*}{}
  
\end{variable}

\begin{variable}{*read-eval*}{}
  
\end{variable}

\begin{variable}{*read-suppress*}{}
  
\end{variable}

\begin{variable}{*readtable*}{}
  
\end{variable}

\section{Printer}

\begin{function}{copy-pprint-dispatch}{\op table}
  
\end{function}

\begin{function}{format}{destination control-string \rest format-arguments}
  Provides various facilities for formatting output.
  CONTROL-STRING contains a string to be output, possibly with embedded
  directives, which are flagged with the escape character "\~{}\%". Directives
  generally expand into additional text to be output, usually consuming one
  or more of the FORMAT-ARGUMENTS in the process. A few useful directives
  are:
        \~{}\%A or \~{}\%nA   Prints one argument as if by PRINC
        \~{}\%S or \~{}\%nS   Prints one argument as if by PRIN1
        \~{}\%D or \~{}\%nD   Prints one argument as a decimal integer
        \~{}\%\%          Does a TERPRI
        \~{}\%\&          Does a FRESH-LINE
  where n is the width of the field in which the object is printed.

  DESTINATION controls where the result will go. If DESTINATION is T, then
  the output is sent to the standard output stream. If it is NIL, then the
  output is returned in a string as the value of the call. Otherwise,
  DESTINATION must be a stream to which the output will be sent.

  Example:   (FORMAT NIL "The answer is \~{}\%D." 10) => "The answer is 10."

  FORMAT has many additional capabilities not described here. Consult the
  manual for details.
\end{function}

\begin{function}{pprint}{object \op stream}
  Prettily output OBJECT preceded by a newline.
\end{function}

\begin{function}{pprint-dispatch}{object \op table}
  
\end{function}

\begin{function}{pprint-fill}{stream list \op colon? atsign?}
  Output LIST to STREAM putting :FILL conditional newlines between each
   element. If COLON? is NIL (defaults to T), then no parens are printed
   around the output. ATSIGN? is ignored (but allowed so that PPRINT-FILL
   can be used with the \~{}\%/.../ format directive.
\end{function}

\begin{function}{pprint-indent}{relative-to n \op stream}
  Specify the indentation to use in the current logical block if
STREAM (which defaults to *STANDARD-OUTPUT*) is a pretty-printing
stream and do nothing if not. (See PPRINT-LOGICAL-BLOCK.) N is the
indentation to use (in ems, the width of an ``m'') and RELATIVE-TO can
be either:

     :BLOCK - Indent relative to the column the current logical block
        started on.

     :CURRENT - Indent relative to the current column.

The new indentation value does not take effect until the following
line break.
\end{function}

\begin{function}{pprint-linear}{stream list \op colon? atsign?}
  Output LIST to STREAM putting :LINEAR conditional newlines between each
   element. If COLON? is NIL (defaults to T), then no parens are printed
   around the output. ATSIGN? is ignored (but allowed so that PPRINT-LINEAR
   can be used with the \~{}\%/.../ format directive.
\end{function}

\begin{function}{pprint-newline}{kind \op stream}
  Output a conditional newline to STREAM (which defaults to
   *STANDARD-OUTPUT*) if it is a pretty-printing stream, and do
   nothing if not. KIND can be one of:
     :LINEAR - A line break is inserted if and only if the immediatly
        containing section cannot be printed on one line.
     :MISER - Same as LINEAR, but only if ``miser-style'' is in effect.
        (See *PRINT-MISER-WIDTH*.)
     :FILL - A line break is inserted if and only if either:
       (a) the following section cannot be printed on the end of the
           current line,
       (b) the preceding section was not printed on a single line, or
       (c) the immediately containing section cannot be printed on one
           line and miser-style is in effect.
     :MANDATORY - A line break is always inserted.
   When a line break is inserted by any type of conditional newline, any
   blanks that immediately precede the conditional newline are ommitted
   from the output and indentation is introduced at the beginning of the
   next line. (See PPRINT-INDENT.)
\end{function}

\begin{function}{pprint-tab}{kind colnum colinc \op stream}
  If STREAM (which defaults to *STANDARD-OUTPUT*) is a pretty-printing
   stream, perform tabbing based on KIND, otherwise do nothing. KIND can
   be one of:
     :LINE - Tab to column COLNUM. If already past COLNUM tab to the next
       multiple of COLINC.
     :SECTION - Same as :LINE, but count from the start of the current
       section, not the start of the line.
     :LINE-RELATIVE - Output COLNUM spaces, then tab to the next multiple of
       COLINC.
     :SECTION-RELATIVE - Same as :LINE-RELATIVE, but count from the start
       of the current section, not the start of the line.
\end{function}

\begin{function}{pprint-tabular}{stream list \op colon? atsign? tabsize}
  Output LIST to STREAM tabbing to the next column that is an even multiple
   of TABSIZE (which defaults to 16) between each element. :FILL style
   conditional newlines are also output between each element. If COLON? is
   NIL (defaults to T), then no parens are printed around the output.
   ATSIGN? is ignored (but allowed so that PPRINT-TABULAR can be used with
   the \~{}\%/.../ format directive.
\end{function}

\begin{function}{prin1}{object \op stream}
  Output a mostly READable printed representation of OBJECT on the specified
  STREAM.
\end{function}

\begin{function}{prin1-to-string}{object}
  Return the printed representation of OBJECT as a string with
   slashification on.
\end{function}

\begin{function}{princ}{object \op stream}
  Output an aesthetic but not necessarily READable printed representation
  of OBJECT on the specified STREAM.
\end{function}

\begin{function}{princ-to-string}{object}
  Return the printed representation of OBJECT as a string with
  slashification off.
\end{function}

\begin{function}{print}{object \op stream}
  Output a newline, the mostly READable printed representation of OBJECT, and
  space to the specified STREAM.
\end{function}

\begin{function}{print-not-readable-object}{condition}
  
\end{function}

\begin{function}{set-pprint-dispatch}{type function \op priority table}
  
\end{function}

\begin{function}{write}{object \key (stream stream) (escape *print-escape*) (radix *print-radix*)
 (base *print-base*) (circle *print-circle*) (pretty *print-pretty*)
 (level *print-level*) (length *print-length*) (case *print-case*)
 (array *print-array*) (gensym *print-gensym*) (readably *print-readably*)
 (right-margin *print-right-margin*) (miser-width *print-miser-width*)
 (lines *print-lines*) (pprint-dispatch *print-pprint-dispatch*)}
  Output OBJECT to the specified stream, defaulting to *STANDARD-OUTPUT*
\end{function}

\begin{function}{write-to-string}{object \key (escape *print-escape*) (radix *print-radix*) (base *print-base*)
 (circle *print-circle*) (pretty *print-pretty*) (level *print-level*)
 (length *print-length*) (case *print-case*) (array *print-array*)
 (gensym *print-gensym*) (readably *print-readably*)
 (right-margin *print-right-margin*) (miser-width *print-miser-width*)
 (lines *print-lines*) (pprint-dispatch *print-pprint-dispatch*)}
  Return the printed representation of OBJECT as a string.
\end{function}

\begin{generic}{print-object}{object stream}
  
\end{generic}

\begin{macro}{formatter}{control-string}
  
\end{macro}

\begin{macro}{pprint-logical-block}{stream-symbol \body body \env env}
  Group some output into a logical block. STREAM-SYMBOL should be either a
   stream, T (for *TERMINAL-IO*), or NIL (for *STANDARD-OUTPUT*). The printer
   control variable *PRINT-LEVEL* is automatically handled.
\end{macro}

\begin{macro}{print-unreadable-object}{object \body body}
  Output OBJECT to STREAM with "\#<" prefix, ">" suffix, optionally
  with object-type prefix and object-identity suffix, and executing the
  code in BODY to provide possible further output.
\end{macro}

\begin{variable}{*print-array*}{}
  
\end{variable}

\begin{variable}{*print-base*}{}
  
\end{variable}

\begin{variable}{*print-case*}{}
  
\end{variable}

\begin{variable}{*print-circle*}{}
  
\end{variable}

\begin{variable}{*print-escape*}{}
  
\end{variable}

\begin{variable}{*print-gensym*}{}
  
\end{variable}

\begin{variable}{*print-length*}{}
  
\end{variable}

\begin{variable}{*print-level*}{}
  
\end{variable}

\begin{variable}{*print-lines*}{}
  
\end{variable}

\begin{variable}{*print-miser-width*}{}
  
\end{variable}

\begin{variable}{*print-pprint-dispatch*}{}
  
\end{variable}

\begin{variable}{*print-pretty*}{}
  
\end{variable}

\begin{variable}{*print-radix*}{}
  
\end{variable}

\begin{variable}{*print-readably*}{}
  
\end{variable}

\begin{variable}{*print-right-margin*}{}
  
\end{variable}

\section{Format}

\section{Streams}

\begin{function}{broadcast-stream-streams}{instance}
  
\end{function}

\begin{function}{clear-input}{\op stream}
  
\end{function}

\begin{function}{clear-output}{\op stream}
  
\end{function}

\begin{function}{close}{stream \key abort}
  Close the given STREAM. No more I/O may be performed, but
  inquiries may still be made. If :ABORT is true, an attempt is made
  to clean up the side effects of having created the stream.
\end{function}

\begin{function}{concatenated-stream-streams}{instance}
  
\end{function}

\begin{function}{echo-stream-input-stream}{instance}
  
\end{function}

\begin{function}{echo-stream-output-stream}{instance}
  
\end{function}

\begin{function}{file-length}{stream}
  
\end{function}

\begin{function}{file-position}{stream \op position}
  
\end{function}

\begin{function}{file-string-length}{stream object}
  
\end{function}

\begin{function}{finish-output}{\op stream}
  
\end{function}

\begin{function}{force-output}{\op stream}
  
\end{function}

\begin{function}{fresh-line}{\op stream}
  
\end{function}

\begin{function}{get-output-stream-string}{stream}
  
\end{function}

\begin{function}{input-stream-p}{stream}
  Can STREAM perform input operations?
\end{function}

\begin{function}{interactive-stream-p}{stream}
  Is STREAM an interactive stream?
\end{function}

\begin{function}{listen}{\op stream}
  
\end{function}

\begin{function}{make-broadcast-stream}{\rest streams}
  
\end{function}

\begin{function}{make-concatenated-stream}{\rest streams}
  Return a stream which takes its input from each of the streams in turn,
   going on to the next at EOF.
\end{function}

\begin{function}{make-echo-stream}{input-stream output-stream}
  Return a bidirectional stream which gets its input from INPUT-STREAM and
   sends its output to OUTPUT-STREAM. In addition, all input is echoed to
   the output stream.
\end{function}

\begin{function}{make-string-input-stream}{string \op start end}
  Return an input stream which will supply the characters of STRING between
  START and END in order.
\end{function}

\begin{function}{make-string-output-stream}{\key element-type \aux buffer}
  Return an output stream which will accumulate all output given it for the
benefit of the function GET-OUTPUT-STREAM-STRING.
\end{function}

\begin{function}{make-synonym-stream}{symbol}
  
\end{function}

\begin{function}{make-two-way-stream}{input-stream output-stream}
  Return a bidirectional stream which gets its input from INPUT-STREAM and
   sends its output to OUTPUT-STREAM.
\end{function}

\begin{function}{open}{filename \key direction element-type if-exists if-does-not-exist
 external-format \aux direction if-does-not-exist if-exists}
  Return a stream which reads from or writes to FILENAME.
  Defined keywords:
   :DIRECTION - one of :INPUT, :OUTPUT, :IO, or :PROBE
   :ELEMENT-TYPE - the type of object to read or write, default BASE-CHAR
   :IF-EXISTS - one of :ERROR, :NEW-VERSION, :RENAME, :RENAME-AND-DELETE,
                       :OVERWRITE, :APPEND, :SUPERSEDE or NIL
   :IF-DOES-NOT-EXIST - one of :ERROR, :CREATE or NIL
  See the manual for details.
\end{function}

\begin{function}{open-stream-p}{stream}
  Return true if STREAM is not closed. A default method is provided
  by class FUNDAMENTAL-STREAM which returns true if CLOSE has not been
  called on the stream.
\end{function}

\begin{function}{output-stream-p}{stream}
  Can STREAM perform output operations?
\end{function}

\begin{function}{peek-char}{\op peek-type stream eof-error-p eof-value recursive-p}
  
\end{function}

\begin{function}{read-byte}{stream \op eof-error-p eof-value}
  
\end{function}

\begin{function}{read-char}{\op stream eof-error-p eof-value recursive-p}
  
\end{function}

\begin{function}{read-char-no-hang}{\op stream eof-error-p eof-value recursive-p}
  
\end{function}

\begin{function}{read-line}{\op stream eof-error-p eof-value recursive-p}
  
\end{function}

\begin{function}{read-sequence}{seq stream \key start end}
  Destructively modify SEQ by reading elements from STREAM.
  That part of SEQ bounded by START and END is destructively modified by
  copying successive elements into it from STREAM. If the end of file
  for STREAM is reached before copying all elements of the subsequence,
  then the extra elements near the end of sequence are not updated, and
  the index of the next element is returned.
\end{function}

\begin{function}{stream-element-type}{stream}
  Return a type specifier for the kind of object returned by the
  STREAM. The class FUNDAMENTAL-CHARACTER-STREAM provides a default method
  which returns CHARACTER.
\end{function}

\begin{function}{stream-error-stream}{condition}
  
\end{function}

\begin{function}{stream-external-format}{stream}
  
\end{function}

\begin{function}{streamp}{stream}
  
\end{function}

\begin{function}{synonym-stream-symbol}{instance}
  
\end{function}

\begin{function}{terpri}{\op stream}
  
\end{function}

\begin{function}{two-way-stream-input-stream}{instance}
  
\end{function}

\begin{function}{two-way-stream-output-stream}{instance}
  
\end{function}

\begin{function}{unread-char}{character \op stream}
  
\end{function}

\begin{function}{write-byte}{integer stream}
  
\end{function}

\begin{function}{write-char}{character \op stream}
  
\end{function}

\begin{function}{write-line}{string \op stream \key start end}
  
\end{function}

\begin{function}{write-sequence}{seq stream \key start end}
  Write the elements of SEQ bounded by START and END to STREAM.
\end{function}

\begin{function}{write-string}{string \op stream \key start end}
  
\end{function}

\begin{function}{y-or-n-p}{\op format-string \rest arguments}
  Y-OR-N-P prints the message, if any, and reads characters from
   *QUERY-IO* until the user enters y or Y as an affirmative, or either
   n or N as a negative answer. It asks again if you enter any other
   characters.
\end{function}

\begin{function}{yes-or-no-p}{\op format-string \rest arguments}
  YES-OR-NO-P is similar to Y-OR-N-P, except that it clears the
   input buffer, beeps, and uses READ-LINE to get the strings
   YES or NO.
\end{function}

\begin{macro}{with-input-from-string}{var \body forms-decls}
  
\end{macro}

\begin{macro}{with-open-file}{stream \body body}
  
\end{macro}

\begin{macro}{with-open-stream}{var \body forms-decls}
  
\end{macro}

\begin{macro}{with-output-to-string}{var \body forms-decls}
  
\end{macro}

\begin{class}{broadcast-stream}{}
  
\end{class}

\begin{class}{concatenated-stream}{}
  
\end{class}

\begin{class}{echo-stream}{}
  
\end{class}

\begin{class}{file-stream}{}
  
\end{class}

\begin{class}{stream}{}
  
\end{class}

\begin{class}{string-stream}{}
  
\end{class}

\begin{class}{synonym-stream}{}
  
\end{class}

\begin{class}{two-way-stream}{}
  
\end{class}

\begin{variable}{*debug-io*}{}
  
\end{variable}

\begin{variable}{*error-output*}{}
  
\end{variable}

\begin{variable}{*query-io*}{}
  
\end{variable}

\begin{variable}{*standard-input*}{}
  
\end{variable}

\begin{variable}{*standard-output*}{}
  
\end{variable}

\begin{variable}{*terminal-io*}{}
  
\end{variable}

\begin{variable}{*trace-output*}{}
  
\end{variable}

\section{Extensions}
\label{sec:extensions}

\begin{macro}{with-input-from-file}{stream-name file-name \rest args
    \key direction \akeys \body body}[cl-ext]
  Evaluate BODY with STREAM-NAME bound to an input-stream from file
FILE-NAME. ARGS is passed directly to open.
\end{macro}

\begin{macro}{with-output-to-file}{stream-name file-name \rest args
    \key direction \akeys \body body}[cl-ext]
  Evaluate BODY with STREAM-NAME to an output stream on the file
FILE-NAME. ARGS is sent as is to the call te open. It will supersed
file-name if it exists.
\begin{devnote}
  The supersed thing is not originaly in alexandria, but I think it's
  more useful.
\end{devnote}
\end{macro}

\begin{function}{write-string-into-file}{string pathname \key if-exists if-does-not-exist external-format}[cl-ext]
  Write STRING to PATHNAME.

The EXTERNAL-FORMAT parameter will be passed to
ENCODING-KEYWORD-TO-NATIVE, see ENCODING-KEYWORD-TO-NATIVE to
possible values.
\end{function}

\begin{function}{read-file-into-string}{pathname \key buffer-size external-format}[cl-ext]
  Return the contents of PATHNAME as a fresh string.

The file specified by PATHNAME will be read one ELEMENT-TYPE
element at a time, the EXTERNAL-FORMAT and ELEMENT-TYPEs must be
compatible.

The EXTERNAL-FORMAT parameter will be passed to
ENCODING-KEYWORD-TO-NATIVE, see ENCODING-KEYWORD-TO-NATIVE to
possible values.
\end{function}

\begin{function}{copy-file}{from to \key if-to-exists element-type}[cl-ext]
  
\end{function}

\begin{function}{copy-stream}{input output \op element-type}[cl-ext]
  Reads data from FROM and writes it to TO. Both FROM and TO must be streams,
they will be passed to read-sequence/write-sequence and must have compatable
element-types.
\end{function}


%%% Local Variables:
%%% mode: latex
%%% TeX-master: cl-dist-manual.tex
%%% End:

\input{cha-file-and-filename}
\input{cha-functions}
\input{cha-macros}
\input{cha-variables}
\input{cha-control-structures}
\input{cha-iteration}
\input{cha-oop}
\input{cha-types}
\input{cha-package}
\chapter{Conditions, restarts and Errors}

\begin{function}{arithmetic-error-operands}{condition}
  
\end{function}

\begin{function}{arithmetic-error-operation}{condition}
  
\end{function}

\begin{function}{abort}{\op condition}
  Transfer control to a restart named ABORT, signalling a CONTROL-ERROR if
   none exists.
\end{function}

\begin{function}{break}{\op datum \rest arguments}
  Print a message and invoke the debugger without allowing any possibility
   of condition handling occurring.
\end{function}

\begin{function}{cell-error-name}{condition}
  
\end{function}

\begin{function}{cerror}{continue-string datum \rest arguments}
  
\end{function}

\begin{function}{compute-restarts}{\op condition}
  Return a list of all the currently active restarts ordered from most recently
established to less recently established. If CONDITION is specified, then only
restarts associated with CONDITION (or with no condition) will be returned.
\end{function}

\begin{function}{continue}{\op condition}
  Transfer control to a restart named CONTINUE, or return NIL if none exists.
\end{function}

\begin{function}{error}{datum \rest arguments}
  Invoke the signal facility on a condition formed from DATUM and ARGUMENTS.
  If the condition is not handled, the debugger is invoked.
\end{function}

\begin{function}{find-restart}{identifier \op condition}
  Return the first restart identified by IDENTIFIER. If IDENTIFIER is a symbol,
then the innermost applicable restart with that name is returned. If IDENTIFIER
is a restart, it is returned if it is currently active. Otherwise NIL is
returned. If CONDITION is specified and not NIL, then only restarts associated
with that condition (or with no condition) will be returned.
\end{function}

\begin{function}{invalid-method-error}{method format-control \rest format-arguments}
  
\end{function}

\begin{function}{invoke-debugger}{condition}
  Enter the debugger.
\end{function}

\begin{function}{invoke-restart}{restart \rest values}
  Calls the function associated with the given restart, passing any given
   arguments. If the argument restart is not a restart or a currently active
   non-nil restart name, then a CONTROL-ERROR is signalled.
\end{function}

\begin{function}{invoke-restart-interactively}{restart}
  Calls the function associated with the given restart, prompting for any
   necessary arguments. If the argument restart is not a restart or a
   currently active non-NIL restart name, then a CONTROL-ERROR is signalled.
\end{function}

\begin{function}{make-condition}{type \rest args}
  Make an instance of a condition object using the specified initargs.
\end{function}

\begin{function}{method-combination-error}{format-control \rest format-arguments}
  
\end{function}

\begin{function}{muffle-warning}{\op condition}
  Transfer control to a restart named MUFFLE-WARNING, signalling a
   CONTROL-ERROR if none exists.
\end{function}

\begin{function}{restart-name}{instance}
  Return the name of the given restart object.
\end{function}

\begin{function}{signal}{datum \rest arguments}
  Invokes the signal facility on a condition formed from DATUM and
   ARGUMENTS. If the condition is not handled, NIL is returned. If
   (TYPEP condition *BREAK-ON-SIGNALS*) is true, the debugger is invoked
   before any signalling is done.
\end{function}

\begin{function}{simple-condition-format-arguments}{condition}
  
\end{function}

\begin{function}{simple-condition-format-control}{condition}
  
\end{function}

\begin{function}{store-value}{value \op condition}
  Transfer control and VALUE to a restart named STORE-VALUE, or return NIL if
   none exists.
\end{function}

\begin{function}{use-value}{value \op condition}
  Transfer control and VALUE to a restart named USE-VALUE, or return NIL if
   none exists.
\end{function}

\begin{function}{warn}{datum \rest arguments}
  Warn about a situation by signalling a condition formed by DATUM and
   ARGUMENTS. While the condition is being signaled, a MUFFLE-WARNING restart
   exists that causes WARN to immediately return NIL.
\end{function}

\begin{macro}{assert}{test-form \op places datum \rest arguments}
  Signals an error if the value of test-form is nil. Continuing from this
   error using the CONTINUE restart will allow the user to alter the value of
   some locations known to SETF, starting over with test-form. Returns NIL.
\end{macro}

\begin{macro}{check-type}{place type \op type-string \env env}
  Signal a restartable error of type TYPE-ERROR if the value of PLACE
is not of the specified type. If an error is signalled and the restart
is used to return, this can only return if the STORE-VALUE restart is
invoked. In that case it will store into PLACE and start over.
\end{macro}

\begin{macro}{define-condition}{name \rest \rest \body options}
  DEFINE-CONDITION Name (Parent-Type*) (Slot-Spec*) Option*
   Define NAME as a condition type. This new type inherits slots and its
   report function from the specified PARENT-TYPEs. A slot spec is a list of:
     (slot-name :reader <rname> :initarg <iname> {Option Value}*

   The DEFINE-CLASS slot options :ALLOCATION, :INITFORM, [slot] :DOCUMENTATION
   and :TYPE and the overall options :DEFAULT-INITARGS and
   [type] :DOCUMENTATION are also allowed.

   The :REPORT option is peculiar to DEFINE-CONDITION. Its argument is either
   a string or a two-argument lambda or function name. If a function, the
   function is called with the condition and stream to report the condition.
   If a string, the string is printed.

   Condition types are classes, but (as allowed by ANSI and not as described in
   CLtL2) are neither STANDARD-OBJECTs nor STRUCTURE-OBJECTs. WITH-SLOTS and
   SLOT-VALUE may not be used on condition objects.
\end{macro}

\begin{macro}{handler-bind}{bindings \body forms}
  (HANDLER-BIND ( {(type handler)}* )  body)

Executes body in a dynamic context where the given handler bindings are in
effect. Each handler must take the condition being signalled as an argument.
The bindings are searched first to last in the event of a signalled
condition.
\end{macro}

\begin{macro}{handler-case}{form \rest cases}
  (HANDLER-CASE form { (type ([var]) body) }* )

Execute FORM in a context with handlers established for the condition types. A
peculiar property allows type to be :NO-ERROR. If such a clause occurs, and
form returns normally, all its values are passed to this clause as if by
MULTIPLE-VALUE-CALL. The :NO-ERROR clause accepts more than one var
specification.
\end{macro}

\begin{macro}{ignore-errors}{\rest forms}
  Execute FORMS handling ERROR conditions, returning the result of the last
  form, or (VALUES NIL the-ERROR-that-was-caught) if an ERROR was handled.
\end{macro}

\begin{macro}{restart-bind}{bindings \body forms}
  Executes forms in a dynamic context where the given restart bindings are
   in effect. Users probably want to use RESTART-CASE. When clauses contain
   the same restart name, FIND-RESTART will find the first such clause.
\end{macro}

\begin{macro}{restart-case}{expression \body clauses \env env}
  (RESTART-CASE form
   {(case-name arg-list {keyword value}* body)}*)
   The form is evaluated in a dynamic context where the clauses have special
   meanings as points to which control may be transferred (see INVOKE-RESTART).
   When clauses contain the same case-name, FIND-RESTART will find the first
   such clause. If Expression is a call to SIGNAL, ERROR, CERROR or WARN (or
   macroexpands into such) then the signalled condition will be associated with
   the new restarts.
\end{macro}

\begin{macro}{with-condition-restarts}{condition-form restarts-form \body body}
  Evaluates the BODY in a dynamic environment where the restarts in the list
   RESTARTS-FORM are associated with the condition returned by CONDITION-FORM.
   This allows FIND-RESTART, etc., to recognize restarts that are not related
   to the error currently being debugged. See also RESTART-CASE.
\end{macro}

\begin{macro}{with-simple-restart}{restart-name \body forms}
  (WITH-SIMPLE-RESTART (restart-name format-string format-arguments)
   body)
   If restart-name is not invoked, then all values returned by forms are
   returned. If control is transferred to this restart, it immediately
   returns the values NIL and T.
\end{macro}

\begin{class}{restart}{}
  
\end{class}

\begin{variable}{*break-on-signals*}{}
  
\end{variable}

\begin{variable}{*debugger-hook*}{}
  
\end{variable}

\section{Extensions}
\label{sec:extensions}

\begin{function}{required-argument}{\op name}[cl-ext]
  Signals an error for a missing argument of NAME. Intended for
use as an initialization form for structure and class-slots, and
a default value for required keyword arguments.
\end{function}

these functions also have conditions with the same name:

\begin{function}{simple-parse-error}{message \rest args}[cl-ext]
  
\end{function}

\begin{function}{simple-program-error}{message \rest args}[cl-ext]
  
\end{function}

\begin{function}{simple-reader-error}{stream message \rest args}[cl-ext]
  
\end{function}

\begin{function}{simple-style-warning}{message \rest args}[cl-ext]
  
\end{function}

\begin{macro}{unwind-protect-case}{\op protected-form \body clauses}[cl-ext]
  Like CL:UNWIND-PROTECT, but you can specify the circumstances that
the cleanup CLAUSES are run.

ABORT-FLAG is the name of a variable that will be bound to T in
CLAUSES if the PROTECTED-FORM aborted preemptively, and to NIL
otherwise.
\end{macro}

\begin{macro}{ignore-some-conditions}{\rest \body body}[cl-ext]
  Similar to CL:IGNORE-ERRORS but the (unevaluated) CONDITIONS
list determines which specific conditions are to be ignored.
\end{macro}

%%% Local Variables:
%%% mode: latex
%%% TeX-master: cl-dist-manual.tex
%%% End:

\input{cha-environment}
\input{cha-debugging}
\chapter{Compilation and evaluation}

\section{Compilation}

\section{Declarations}

\begin{accessor}{compiler-macro-function}{name \op env}
  If NAME names a compiler-macro in ENV, return the expansion function, else
return NIL. Can be set with SETF when ENV is NIL.
\end{accessor}

\begin{accessor}{macro-function}{symbol \op env}
  If SYMBOL names a macro in ENV, returns the expansion function,
else returns NIL. If ENV is unspecified or NIL, use the global environment
only.
\end{accessor}

\begin{function}{compile}{name \op definition}
  Coerce DEFINITION (by default, the function whose name is NAME)
  to a compiled function, returning (VALUES THING WARNINGS-P FAILURE-P),
  where if NAME is NIL, THING is the result of compilation, and
  otherwise THING is NAME. When NAME is not NIL, the compiled function
  is also set into (MACRO-FUNCTION NAME) if NAME names a macro, or into
  (FDEFINITION NAME) otherwise.
\end{function}

\begin{function}{constantp}{form \op environment}
  True of any FORM that has a constant value: self-evaluating objects,
keywords, defined constants, quote forms. Additionally the
constant-foldability of some function calls special forms is recognized. If
ENVIRONMENT is provided the FORM is first macroexpanded in it.
\end{function}

\begin{function}{eval}{original-exp}
  Evaluate the argument in a null lexical environment, returning the
   result or results.
\end{function}

\begin{function}{macroexpand}{form \op env}
  Repetitively call MACROEXPAND-1 until the form can no longer be expanded.
   Returns the final resultant form, and T if it was expanded. ENV is the
   lexical environment to expand in, or NIL (the default) for the null
   environment.
\end{function}

\begin{function}{macroexpand-1}{form \op env}
  If form is a macro (or symbol macro), expand it once. Return two values,
   the expanded form and a T-or-NIL flag indicating whether the form was, in
   fact, a macro. ENV is the lexical environment to expand in, which defaults
   to the null environment.
\end{function}

\begin{function}{proclaim}{raw-form}
  
\end{function}

\begin{function}{special-operator-p}{symbol}
  If the symbol globally names a special form, return T, otherwise NIL.
\end{function}

\begin{macro}{declaim}{\rest specs}
  DECLAIM Declaration*
  Do a declaration or declarations for the global environment.
\end{macro}

\begin{macro}{define-compiler-macro}{name lambda-list \body body}
  Define a compiler-macro for NAME.
\end{macro}

\begin{macro}{define-symbol-macro}{name expansion}
  
\end{macro}

\begin{macro}{defmacro}{name lambda-list \rest body}
  
\end{macro}

\begin{macro}{lambda}{\whole whole args \body body}
  
\end{macro}

\begin{specialop}{eval-when}{situations \rest forms}
  EVAL-WHEN (situation*) form*

Evaluate the FORMS in the specified SITUATIONS (any of :COMPILE-TOPLEVEL,
:LOAD-TOPLEVEL, or :EXECUTE, or (deprecated) COMPILE, LOAD, or EVAL).
\end{specialop}

\begin{specialop}{load-time-value}{form \op read-only-p}
  Arrange for FORM to be evaluated at load-time and use the value produced
   as if it were a constant. If READ-ONLY-P is non-NIL, then the resultant
   object is guaranteed to never be modified, so it can be put in read-only
   storage.
\end{specialop}

\begin{specialop}{locally}{\body body}
  LOCALLY declaration* form*

Sequentially evaluate the FORMS in a lexical environment where the
DECLARATIONS have effect. If LOCALLY is a top level form, then the FORMS are
also processed as top level forms.
\end{specialop}

\begin{specialop}{quote}{thing}
  QUOTE value

Return VALUE without evaluating it.
\end{specialop}

\begin{specialop}{symbol-macrolet}{macrobindings \body body}
  SYMBOL-MACROLET ({(name expansion)}*) decl* form*

Define the NAMES as symbol macros with the given EXPANSIONS. Within the
body, references to a NAME will effectively be replaced with the EXPANSION.
\end{specialop}

\begin{specialop}{the}{value-type form}
  Specifies that the values returned by FORM conform to the VALUE-TYPE.

CLHS specifies that the consequences are undefined if any result is
not of the declared type, but SBCL treats declarations as assertions
as long as SAFETY is at least 2, in which case incorrect type
information will result in a runtime type-error instead of leading to
eg. heap corruption. This is however expressly non-portable: use
CHECK-TYPE instead of THE to catch type-errors at runtime. THE is best
considered an optimization tool to inform the compiler about types it
is unable to derive from other declared types.
\end{specialop}

\begin{variable}{*macroexpand-hook*}{}
  
\end{variable}

\section{Extensions}
\label{sec:extensions}

\begin{macro}{eval-always}{\body body}[cl-ext]
  
\end{macro}


%%% Local Variables:
%%% mode: latex
%%% TeX-master: cl-dist-manual.tex
%%% End:

\input{cha-pitfalls}

\part{Extension Libraries}
\label{part:libraries}

\chapter{HTML and XML processing}

\chapter{GUI programming}

\chapter{Threads}

\chapter{Data persistance}

\chapter{Data Compression and Archiving}

\chapter{Generic Operating System Services}

time, parse options, logging, password, ffi

\chapter{File Formats}

\chapter{Cryptographic Services}

\chapter{Internationalization}

\chapter{Development Tools}

\section{Testing}

\section{Automatic documentation}

\chapter{Internet Data Handling}

email, json, mailbox, mime, etc

\printindex

\end{document}

%%% Local Variables:
%%% mode: latex
%%% TeX-master: cl-dist-manual.tex
%%% End:
