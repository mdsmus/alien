\section{Files}

\begin{function}{delete-file}{file}
  Delete the specified FILE.
\end{function}

\begin{function}{directory}{pathname \key resolve-symlinks}
  Return a list of PATHNAMEs, each the TRUENAME of a file that matched the
   given pathname. Note that the interaction between this ANSI-specified
   TRUENAMEing and the semantics of the Unix filesystem (symbolic links..)
   means this function can sometimes return files which don't have the same
   directory as PATHNAME.  If :RESOLVE-SYMLINKS is NIL, don't resolve
   symbolic links in matching filenames.
\end{function}

\begin{function}{ensure-directories-exist}{pathspec \key verbose mode}
  Test whether the directories containing the specified file
  actually exist, and attempt to create them if they do not.
  The MODE argument is a CMUCL/SBCL-specific extension to control
  the Unix permission bits.
\end{function}

\begin{function}{file-author}{pathspec}
  Return the author of the file specified by PATHSPEC. Signal an
error of type FILE-ERROR if no such file exists, or if PATHSPEC
is a wild pathname.
\end{function}

\begin{function}{file-error-pathname}{condition}
  
\end{function}

\begin{function}{file-write-date}{pathspec}
  Return the write date of the file specified by PATHSPEC.
An error of type FILE-ERROR is signaled if no such file exists,
or if PATHSPEC is a wild pathname.
\end{function}

\begin{function}{probe-file}{pathspec}
  Return the truename of PATHSPEC if the truename can be found,
or NIL otherwise.  See TRUENAME for more information.
\end{function}

\begin{function}{rename-file}{file new-name}
  Rename FILE to have the specified NEW-NAME. If FILE is a stream open to a
  file, then the associated file is renamed.
\end{function}

\begin{function}{truename}{pathspec}
  If PATHSPEC is a pathname that names an existing file, return
a pathname that denotes a canonicalized name for the file.  If
pathspec is a stream associated with a file, return a pathname
that denotes a canonicalized name for the file associated with
the stream.

An error of type FILE-ERROR is signalled if no such file exists
or if the file system is such that a canonicalized file name
cannot be determined or if the pathname is wild.

Under Unix, the TRUENAME of a symlink that links to itself or to
a file that doesn't exist is considered to be the name of the
broken symlink itself.
\end{function}

\section{Filenames}

\begin{accessor}{logical-pathname-translations}{host}
  Return the (logical) host object argument's list of translations.
\end{accessor}

\begin{function}{directory-namestring}{pathname}
  Return a string representation of the directories used in the pathname.
\end{function}

\begin{function}{enough-namestring}{pathname \op defaults}
  Return an abbreviated pathname sufficent to identify the pathname relative
   to the defaults.
\end{function}

\begin{function}{file-namestring}{pathname}
  Return a string representation of the name used in the pathname.
\end{function}

\begin{function}{host-namestring}{pathname}
  Return a string representation of the name of the host in the pathname.
\end{function}

\begin{function}{load-logical-pathname-translations}{host}
  
\end{function}

\begin{function}{logical-pathname}{pathspec}
  Converts the pathspec argument to a logical-pathname and returns it.
\end{function}

\begin{function}{make-pathname}{\key host device directory name type version defaults case}
  Makes a new pathname from the component arguments. Note that host is
a host-structure or string.
\end{function}

\begin{function}{merge-pathnames}{pathname \op defaults default-version}
  Construct a filled in pathname by completing the unspecified components
   from the defaults.
\end{function}

\begin{function}{namestring}{pathname}
  Construct the full (name)string form of the pathname.
\end{function}

\begin{function}{parse-namestring}{thing \op host defaults \key start end junk-allowed}
  
\end{function}

\begin{function}{pathname}{pathspec}
  Convert PATHSPEC (a pathname designator) into a pathname.
\end{function}

\begin{function}{pathname-device}{pathname \key case}
  Return PATHNAME's device.
\end{function}

\begin{function}{pathname-directory}{pathname \key case}
  Return PATHNAME's directory.
\end{function}

\begin{function}{pathname-host}{pathname \key case}
  Return PATHNAME's host.
\end{function}

\begin{function}{pathname-match-p}{in-pathname in-wildname}
  Pathname matches the wildname template?
\end{function}

\begin{function}{pathname-name}{pathname \key case}
  Return PATHNAME's name.
\end{function}

\begin{function}{pathname-type}{pathname \key case}
  Return PATHNAME's type.
\end{function}

\begin{function}{pathname-version}{pathname}
  Return PATHNAME's version.
\end{function}

\begin{function}{pathnamep}{object}
  
\end{function}

\begin{function}{translate-logical-pathname}{pathname \key}
  Translate PATHNAME to a physical pathname, which is returned.
\end{function}

\begin{function}{translate-pathname}{source from-wildname to-wildname \key}
  Use the source pathname to translate the from-wildname's wild and
   unspecified elements into a completed to-pathname based on the to-wildname.
\end{function}

\begin{function}{wild-pathname-p}{pathname \op field-key}
  Predicate for determining whether pathname contains any wildcards.
\end{function}

\begin{class}{logical-pathname}{pathspec}
  Converts the pathspec argument to a logical-pathname and returns it.
\end{class}

\begin{class}{pathname}{pathspec}
  Convert PATHSPEC (a pathname designator) into a pathname.
\end{class}

\begin{variable}{*default-pathname-defaults*}{}
  
\end{variable}

%%% Local Variables:
%%% mode: latex
%%% TeX-master: cl-dist-manual.tex
%%% End:
