\chapter{Arrays}
\label{cha:arrays}

\begin{accessor}{aref}{array \rest subscripts}
  Return the element of the ARRAY specified by the SUBSCRIPTS.
\end{accessor}

\begin{accessor}{bit}{bit-array \rest subscripts}
  Return the bit from the BIT-ARRAY at the specified SUBSCRIPTS.
\end{accessor}

\begin{accessor}{fill-pointer}{vector}
  Return the FILL-POINTER of the given VECTOR.
\end{accessor}

\begin{accessor}{row-major-aref}{array index}
  Return the element of array corressponding to the row-major index. This is
   SETF'able.
\end{accessor}

\begin{accessor}{sbit}{simple-bit-array \rest subscripts}
  Return the bit from SIMPLE-BIT-ARRAY at the specified SUBSCRIPTS.
\end{accessor}

\begin{accessor}{svref}{simple-vector index}
  Return the INDEX'th element of the given Simple-Vector.
\end{accessor}

\begin{constant}{array-dimension-limit}{}
  
\end{constant}

\begin{constant}{array-rank-limit}{}
  
\end{constant}

\begin{constant}{array-total-size-limit}{}
  
\end{constant}

\begin{function}{adjust-array}{array dimensions \key element-type initial-element initial-contents
 fill-pointer displaced-to displaced-index-offset}
  Adjust ARRAY's dimensions to the given DIMENSIONS and stuff.
\end{function}

\begin{function}{adjustable-array-p}{array}
  Return T if (ADJUST-ARRAY ARRAY...) would return an array identical
   to the argument, this happens for complex arrays.
\end{function}

\begin{function}{array-dimension}{array axis-number}
  Return the length of dimension AXIS-NUMBER of ARRAY.
\end{function}

\begin{function}{array-dimensions}{array}
  Return a list whose elements are the dimensions of the array
\end{function}

\begin{function}{array-displacement}{array}
  Return the values of :DISPLACED-TO and :DISPLACED-INDEX-offset
   options to MAKE-ARRAY, or NIL and 0 if not a displaced array.
\end{function}

\begin{function}{array-element-type}{array}
  Return the type of the elements of the array
\end{function}

\begin{function}{array-has-fill-pointer-p}{array}
  Return T if the given ARRAY has a fill pointer, or NIL otherwise.
\end{function}

\begin{function}{array-in-bounds-p}{array \rest subscripts}
  Return T if the SUBSCIPTS are in bounds for the ARRAY, NIL otherwise.
\end{function}

\begin{function}{array-rank}{array}
  Return the number of dimensions of ARRAY.
\end{function}

\begin{function}{array-row-major-index}{array \rest subscripts}
  
\end{function}

\begin{function}{array-total-size}{array}
  Return the total number of elements in the Array.
\end{function}

\begin{function}{arrayp}{object}
  Return true if OBJECT is an ARRAY, and NIL otherwise.
\end{function}

\begin{function}{bit-and}{bit-array-1 bit-array-2 \op result-bit-array}
  Perform a bit-wise LOGAND on the elements of BIT-ARRAY-1 and BIT-ARRAY-2,
  putting the results in RESULT-BIT-ARRAY. If RESULT-BIT-ARRAY is T,
  BIT-ARRAY-1 is used. If RESULT-BIT-ARRAY is NIL or omitted, a new array is
  created. All the arrays must have the same rank and dimensions.
\end{function}

\begin{function}{bit-andc1}{bit-array-1 bit-array-2 \op result-bit-array}
  Perform a bit-wise LOGANDC1 on the elements of BIT-ARRAY-1 and BIT-ARRAY-2,
  putting the results in RESULT-BIT-ARRAY. If RESULT-BIT-ARRAY is T,
  BIT-ARRAY-1 is used. If RESULT-BIT-ARRAY is NIL or omitted, a new array is
  created. All the arrays must have the same rank and dimensions.
\end{function}

\begin{function}{bit-andc2}{bit-array-1 bit-array-2 \op result-bit-array}
  Perform a bit-wise LOGANDC2 on the elements of BIT-ARRAY-1 and BIT-ARRAY-2,
  putting the results in RESULT-BIT-ARRAY. If RESULT-BIT-ARRAY is T,
  BIT-ARRAY-1 is used. If RESULT-BIT-ARRAY is NIL or omitted, a new array is
  created. All the arrays must have the same rank and dimensions.
\end{function}

\begin{function}{bit-eqv}{bit-array-1 bit-array-2 \op result-bit-array}
  Perform a bit-wise LOGEQV on the elements of BIT-ARRAY-1 and BIT-ARRAY-2,
  putting the results in RESULT-BIT-ARRAY. If RESULT-BIT-ARRAY is T,
  BIT-ARRAY-1 is used. If RESULT-BIT-ARRAY is NIL or omitted, a new array is
  created. All the arrays must have the same rank and dimensions.
\end{function}

\begin{function}{bit-ior}{bit-array-1 bit-array-2 \op result-bit-array}
  Perform a bit-wise LOGIOR on the elements of BIT-ARRAY-1 and BIT-ARRAY-2,
  putting the results in RESULT-BIT-ARRAY. If RESULT-BIT-ARRAY is T,
  BIT-ARRAY-1 is used. If RESULT-BIT-ARRAY is NIL or omitted, a new array is
  created. All the arrays must have the same rank and dimensions.
\end{function}

\begin{function}{bit-nand}{bit-array-1 bit-array-2 \op result-bit-array}
  Perform a bit-wise LOGNAND on the elements of BIT-ARRAY-1 and BIT-ARRAY-2,
  putting the results in RESULT-BIT-ARRAY. If RESULT-BIT-ARRAY is T,
  BIT-ARRAY-1 is used. If RESULT-BIT-ARRAY is NIL or omitted, a new array is
  created. All the arrays must have the same rank and dimensions.
\end{function}

\begin{function}{bit-nor}{bit-array-1 bit-array-2 \op result-bit-array}
  Perform a bit-wise LOGNOR on the elements of BIT-ARRAY-1 and BIT-ARRAY-2,
  putting the results in RESULT-BIT-ARRAY. If RESULT-BIT-ARRAY is T,
  BIT-ARRAY-1 is used. If RESULT-BIT-ARRAY is NIL or omitted, a new array is
  created. All the arrays must have the same rank and dimensions.
\end{function}

\begin{function}{bit-not}{bit-array \op result-bit-array}
  Performs a bit-wise logical NOT on the elements of BIT-ARRAY,
  putting the results in RESULT-BIT-ARRAY. If RESULT-BIT-ARRAY is T,
  BIT-ARRAY is used. If RESULT-BIT-ARRAY is NIL or omitted, a new array is
  created. Both arrays must have the same rank and dimensions.
\end{function}

\begin{function}{bit-orc1}{bit-array-1 bit-array-2 \op result-bit-array}
  Perform a bit-wise LOGORC1 on the elements of BIT-ARRAY-1 and BIT-ARRAY-2,
  putting the results in RESULT-BIT-ARRAY. If RESULT-BIT-ARRAY is T,
  BIT-ARRAY-1 is used. If RESULT-BIT-ARRAY is NIL or omitted, a new array is
  created. All the arrays must have the same rank and dimensions.
\end{function}

\begin{function}{bit-orc2}{bit-array-1 bit-array-2 \op result-bit-array}
  Perform a bit-wise LOGORC2 on the elements of BIT-ARRAY-1 and BIT-ARRAY-2,
  putting the results in RESULT-BIT-ARRAY. If RESULT-BIT-ARRAY is T,
  BIT-ARRAY-1 is used. If RESULT-BIT-ARRAY is NIL or omitted, a new array is
  created. All the arrays must have the same rank and dimensions.
\end{function}

\begin{function}{bit-vector-p}{object}
  Return true if OBJECT is a BIT-VECTOR, and NIL otherwise.
\end{function}

\begin{function}{bit-xor}{bit-array-1 bit-array-2 \op result-bit-array}
  Perform a bit-wise LOGXOR on the elements of BIT-ARRAY-1 and BIT-ARRAY-2,
  putting the results in RESULT-BIT-ARRAY. If RESULT-BIT-ARRAY is T,
  BIT-ARRAY-1 is used. If RESULT-BIT-ARRAY is NIL or omitted, a new array is
  created. All the arrays must have the same rank and dimensions.
\end{function}

\begin{function}{make-array}{dimensions \key element-type initial-element initial-contents adjustable
 fill-pointer displaced-to displaced-index-offset}
  
\end{function}

\begin{function}{simple-bit-vector-p}{object}
  Return true if OBJECT is a SIMPLE-BIT-VECTOR, and NIL otherwise.
\end{function}

\begin{function}{simple-vector-p}{object}
  Return true if OBJECT is a SIMPLE-VECTOR, and NIL otherwise.
\end{function}

\begin{function}{upgraded-array-element-type}{spec \op environment}
  Return the element type that will actually be used to implement an array
   with the specifier :ELEMENT-TYPE Spec.
\end{function}

\begin{function}{vector}{\rest objects}
  Construct a SIMPLE-VECTOR from the given objects.
\end{function}

\begin{function}{vector-pop}{array}
  Decrease the fill pointer by 1 and return the element pointed to by the
  new fill pointer.
\end{function}

\begin{function}{vector-push}{new-el array}
  Attempt to set the element of ARRAY designated by its fill pointer
   to NEW-EL, and increment the fill pointer by one. If the fill pointer is
   too large, NIL is returned, otherwise the index of the pushed element is
   returned.
\end{function}

\begin{function}{vector-push-extend}{new-element vector \op min-extension}
  
\end{function}

\begin{function}{vectorp}{object}
  Return true if OBJECT is a VECTOR, and NIL otherwise.
\end{function}

\begin{class}{array}{}
  
\end{class}

\begin{class}{bit-vector}{}
  
\end{class}

\begin{class}{vector}{\rest objects}
  Construct a SIMPLE-VECTOR from the given objects.
\end{class}

\section{Extensions}
\label{sec:extensions}

\begin{function}{copy-array}{array \key element-type fill-pointer adjustable}[cl-ext]
  Returns an undisplaced copy of ARRAY, with same fill-pointer
and adjustability (if any) as the original, unless overridden by
the keyword arguments.
\end{function}

\begin{function}{make-displaced-array}{array \op start end}[cl-ext]
  
\end{function}

